% Chapter Chapter 7 For Reproducible Research in R and RStudio
% Christopher Gandrud
% Created: 16/07/2012 05:45:03 pm CEST
% Updated: 15 February 2013




\chapter{Preparing Data for Analysis}\label{DataClean}

Once we have gathered the raw data that we want to include in our statistical analyses we generally need to clean it up so that it can be merged it into a single data file. In this chapter we will learn how to create the data gather and merging files we saw last chapter. The chapter also includes information on recoding and transforming variables. This is important for merging data, but will be very useful information in later chapters as well. If you are very familiar with data transformations in R you may want to skip onto the next chapter. 

\section{Cleaning data for merging}

In order to successfully merge two or more data frames we need to make sure that they are in the same format. Let's look at some of the important formatting issues and how to reformat your data frames so that they can be easily merged.

\subsection{Get a handle on your data}

Before doing anything to your data it is a good idea to take a look at it and see what needs to be done. Taking a little time to become acquainted with your data will help you avoid many error messages and much frustration. 

You could of course just type a data frame object's name into the R console. This will print the entire data frame in your console. For data frames with more than a few variables and observations this is very impractical. We have already seen a number of commands that are useful for looking at parts of your data. As we saw in Chapter \ref{GettingStartedRKnitr}, the \texttt{names}\index{R command!names} command shows you the variable names in a data frame object. The \texttt{head}\index{R command!head} command shows the names plus the first few observations in a data frame. The \texttt{tail}\index{R command!tail} shows the last few. 

Use the \texttt{dim}\index{R command!dim} (dimensions) command to quickly see the number of observations and variables (the number of rows and columns) in a data frame object. For example, let's us the \emph{FertConsumpData} object we created in Chapter \ref{DataGather} to test out \texttt{dim}:

\begin{knitrout}
\definecolor{shadecolor}{rgb}{0.969, 0.969, 0.969}\color{fgcolor}\begin{kframe}
\begin{alltt}
\hlfunctioncall{dim}(FertConsumpData)
\end{alltt}
\begin{verbatim}
## [1] 984   4
\end{verbatim}
\end{kframe}
\end{knitrout}


\noindent The first number is the number of rows in the data frame (984) and the second is the number of columns (4). You can alse use the \texttt{nrow} command to find just the number of rows and \texttt{ncol} to see only the columns.\index{R command!nrow}\index{R command!ncol}

The \texttt{summary} command\index{R command!summary} is especially helpful for seeing basic descriptive statistics for all of the variables in a data frame and also the variables' types. Here is an example:

{\small
\begin{knitrout}
\definecolor{shadecolor}{rgb}{0.969, 0.969, 0.969}\color{fgcolor}\begin{kframe}
\begin{alltt}
\hlcomment{# Summarize FertConsumpData data frame object}
\hlfunctioncall{summary}(FertConsumpData)
\end{alltt}
\begin{verbatim}
##     iso2c             country          AG.CON.FERT.ZS      year     
##  Length:984         Length:984         Min.   :   0   Min.   :2002  
##  Class :character   Class :character   1st Qu.:  12   1st Qu.:2003  
##  Mode  :character   Mode  :character   Median :  80   Median :2004  
##                                        Mean   : 180   Mean   :2004  
##                                        3rd Qu.: 161   3rd Qu.:2004  
##                                        Max.   :8964   Max.   :2005  
##                                        NA's   :251
\end{verbatim}
\end{kframe}
\end{knitrout}

}

\noindent We can immediately see that the variables \textbf{iso2c} and \textbf{country} are character strings. Because \texttt{summary} is able to calculate means, medians, and so on for \textbf{AG.CON.FERT.ZS} and \textbf{year} we know they are numeric. Have a look over the summary to see if there is anything unexpected like lots of missing values (\textbf{NA's}) or unusual maximum and minimum values. You can of course run \texttt{summary} on a particular variable by using the component selector (\verb|$|):

\begin{knitrout}
\definecolor{shadecolor}{rgb}{0.969, 0.969, 0.969}\color{fgcolor}\begin{kframe}
\begin{alltt}
\hlcomment{# Summarize the methane emissions variable from FertConsumpData}
\hlfunctioncall{summary}(FertConsumpData$AG.CON.FERT.ZS)
\end{alltt}
\begin{verbatim}
##    Min. 1st Qu.  Median    Mean 3rd Qu.    Max.    NA's 
##       0      12      80     180     161    8960     251
\end{verbatim}
\end{kframe}
\end{knitrout}


\noindent We'll come back to why knowing this type of information is important for merging and data analysis later in this Chapter.

Another important command for quickly summarizing a data frame is \texttt{table}\index{R command!table} This creates a contingency table\index{contingency table} with counts of the number of observations per combination of factor variables.

You can view a portion of a data frame object with the \texttt{View} command.\index{R command!View} This will open a new window that lets you see a selection of the data frame. If you are using RStudio, you can click on the data frame in the \emph{Workspace} tab\index{RStudio!Workspace tab} and you will get something similar. Note that neither of these viewers are interactive in that you can't use them to manipulate the data. They are only data viewers. To be able to see similar windows that you can interactively edit use the \texttt{fix}\index{R command!fix} command in the same way that you use \texttt{View}. This can be useful for small edits, but remember that the edits are not reproducible.

\subsection{Reshaping Data}\index{R!reshaping data}

Obviously it is usually a good if your data sets are kept in data frame type objects. See Chapter \ref{GettingStartedRKnitr} (page \pageref{data.frame}) for how to convert objects into data frames with the \texttt{data.frame} command.\index{R command!data.frame}\index{R!data frame} Not only do data sets (generally) need to be stored in data frame objects they also need to have the same layout before they can be merged. Most R statistical analysis tools assume that your data is in ``long'' format\index{long formatted data}. This usually means that data frame columns are variables and rows are specific observations (see Table \ref{ExampleLong}).

\begin{table}[h!]
    \caption{Long Formatted Data Example}
    \label{ExampleLong}
    \begin{tabular}{l c}
        \\[0.15cm]
        \hline
        Subject & Variable1 \\
        \hline \\[0.1cm]
        Subject1 & \\[0.25cm]
        Subject2 & \\[0.25cm]
        Subject3 & \\[0.25cm]
        \ldots & \\[0.25cm]
        \hline
    \end{tabular}
\end{table}

\noindent In this chapter we will mostly use examples of time-series cross-sectional data (TSCS)\index{time-series cross-sectional}\index{TSCS} that we want to have in long-format. Long formatted TSCS data is simply a data frame where rows identify observations of a particular subject at particular points in time (see Table \ref{ExampleTSCSLong})

 \begin{table}[h!]
    \caption{Long Formatted Time-series Cross-sectional Data Example}
    \label{ExampleTSCSLong}
    \begin{tabular}{l c c}
        \\[0.15cm]
        \hline
        Subject & Time & Variable1 \\
        \hline \\[0.1cm]
        Subject1 & 1 & \\[0.25cm]
        Subject1 & 2 & \\[0.25cm]
        Subject1 & 3 & \\[0.25cm]
        Subject2 & 1 & \\[0.25cm]
        Subject2 & 2 & \\[0.25cm]
        Subject2 & 3 & \\[0.25cm]
        \ldots & & \\[0.25cm]
        \hline
    \end{tabular}
\end{table}

\noindent In this chapter our TSCS data is specifically going to be countries that are observed in multiple years.

If one of our raw data sets is not in this format then we will need to reshape\index{reshape data} it. Some data sets are in ``wide'' format;\index{wide formatted data} where one of the columns in long formatted data is widened to cover multiple columns. This is confusing to visualize without an example. Table \ref{ExampleWide} shows how Table \ref{ExampleTSCSLong} looks when we widen the time variable.

\begin{table}[h!]
    \caption{Wide Formatted Data Example}
    \label{ExampleWide}
    \begin{tabular}{l c c c}
        \\[0.15cm]
        \hline 
        Subject & Time1 & Time2 & Time3 \\
        \hline \\[0.1cm]
        Subject1 & & & \\[0.25cm]
        Subject2 & & & \\[0.25cm]
        \ldots & & & \\[0.25cm]
        \hline
    \end{tabular}
\end{table}

Reshaping data is often the cause of much confusion and frustration. Though probably never easy, there are a number of useful R functions for changing data from wide format to long and vice versa. These include the matrix transpose command (\textbf{t})\footnote{See this example by Rob Kabacoff: \url{http://www.statmethods.net/management/reshape.html}. Note also that because the matrix transpose function is denoted with \texttt{t}, you should not give any object the name \emph{t}.}\index{matrix transpose} and the \texttt{reshape}\index{R command!reshape} command, both are loaded in R by default. \emph{reshape2} is a very helpful package for reshaping data \citep{R-reshape2}.\index{reshape2}\footnote{Note: confusingly the \emph{reshape2} package does not include the \texttt{reshape} command. The \texttt{reshape} command is part of R's built in \emph{stats} package.\index{stats} I don't cover that command here, because it is less flexible than what \emph{reshape2} can do.} This provides more general tools for reshaping data and is worth investing some time to learn well. In this section we will cover some of \emph{reshape2}'s basic commands and use them to reshape a TSCS data frame from wide to long format. We will also encounter this package again in Chapter \ref{FiguresChapter} when we want to transform data so that it can be graphed.

Let's imagine that the fertilizer consumption data we previously downloaded from the World Bank is in wide rather than long format and is in a data frame objected called \emph{WideFert}. It looks like this:\footnote{Please see the Appendix (page \pageref{WideAppendix}) for the code I used to reshape the data from wide to long format.}







\begin{knitrout}
\definecolor{shadecolor}{rgb}{0.969, 0.969, 0.969}\color{fgcolor}\begin{kframe}
\begin{alltt}
\hlfunctioncall{head}(WideFert)
\end{alltt}
\begin{verbatim}
##    iso2c        country   2002   2003    2004    2005
## 8     AF    Afghanistan  3.403  3.275   4.536   4.240
## 10    AL        Albania 97.185 98.933 100.599 111.597
## 58    DZ        Algeria  9.642  6.002  25.095   7.430
## 14    AS American Samoa     NA     NA      NA      NA
## 6     AD        Andorra     NA     NA      NA      NA
## 12    AO         Angola  1.659  1.789   4.502   2.261
\end{verbatim}
\end{kframe}
\end{knitrout}


\noindent We can use \emph{reshape2}'s \texttt{melt}\index{melt}\index{R command!melt}\label{MeltReshape} command to reshape this data from wide to long format. The term ``melt'' is intended to evoke an image of the data melting down from a wide to long format.\footnote{The opposite \texttt{cast} command (\texttt{dcast}\index{cast}\index{R command!dcast}\index{dcast}\index{R command!dcast} in the case of data frames) is supposed to evoke an image of casting out the data from long to wide format. See page \pageref{WideAppendix} for an example using the \texttt{dcast} command.} In our \emph{WideFert} data we don't want the \textbf{iso2c} and \textbf{country} variables to be melted. These variables identify the data set's subjects. We can tell \texttt{melt} that they are id variables with the \texttt{id.vars} argument. The remaining columns (i.e. \textbf{2002}, \textbf{2003}, \textbf{2004} and \textbf{2005}) will be melted into two new variables: \textbf{variable}, and \textbf{value}. The former will contain the years and the later will contain the fertilizer consumption data. Here is the full code:

\begin{knitrout}
\definecolor{shadecolor}{rgb}{0.969, 0.969, 0.969}\color{fgcolor}\begin{kframe}
\begin{alltt}
\hlcomment{# Melt WideFert}
MoltenFert <- \hlfunctioncall{melt}(data = WideFert, 
                    id.vars = \hlfunctioncall{c}(\hlstring{"iso2c"}, \hlstring{"country"}))

\hlcomment{# Show MoltenFert}
\hlfunctioncall{head}(MoltenFert)
\end{alltt}
\begin{verbatim}
##   iso2c        country variable  value
## 1    AF    Afghanistan     2002  3.403
## 2    AL        Albania     2002 97.185
## 3    DZ        Algeria     2002  9.642
## 4    AS American Samoa     2002     NA
## 5    AD        Andorra     2002     NA
## 6    AO         Angola     2002  1.659
\end{verbatim}
\end{kframe}
\end{knitrout}


\noindent Objects created by \texttt{melt} are often referred to as ``molten'' data in the \emph{reshape2} documentation. That is why I've given our new data frame the name \emph{MoltenFert}. 

\subsection{Renaming variables}\index{R!renaming variables}

Frequently, in the data clean up process we want to change the names of our variables. This will make our data easier to understand and may even be necessary to properly combine data sets (see below). In the previous example, for instance, our \emph{MoltenFert} data frame has two variables--\textbf{variable} and \textbf{value}--that would be easier to understand if they were renamed \textbf{year} and \textbf{FertilizerConsumption}. Renaming data frame variables is straightforward with the \texttt{rename}\index{R command!rename} command in the \emph{plyr} package \citep{R-plyr}.

To rename both \textbf{variable} and \textbf{value} with the \texttt{rename} command type:

\begin{knitrout}
\definecolor{shadecolor}{rgb}{0.969, 0.969, 0.969}\color{fgcolor}\begin{kframe}
\begin{alltt}
\hlcomment{# Load reshape package}
\hlfunctioncall{library}(plyr)

\hlcomment{# Rename variable = year, value = FertilizerConsumption}
MoltenFert <- \hlfunctioncall{rename}(x = MoltenFert,
                     replace = \hlfunctioncall{c}(\hlstring{"variable"} = \hlstring{"year"},
                                 \hlstring{"value"} = \hlstring{"FertilizerConsumption"}))
\end{alltt}


{\ttfamily\noindent\color{warningcolor}{\#\# Warning: unused name(s) selected}}\begin{alltt}

\hlcomment{# Show MoltenFert}
\hlfunctioncall{head}(MoltenFert)
\end{alltt}
\begin{verbatim}
##   iso2c        country variable  value
## 1    AF    Afghanistan     2002  3.403
## 2    AL        Albania     2002 97.185
## 3    DZ        Algeria     2002  9.642
## 4    AS American Samoa     2002     NA
## 5    AD        Andorra     2002     NA
## 6    AO         Angola     2002  1.659
\end{verbatim}
\end{kframe}
\end{knitrout}


\subsection{Ordering data}\index{R!ordering data}

You may have noticed that as a result of melting \emph{WideFert} the data is now ordered by year then country name. Typically TSCS data is sorted by country then year, or more generally: subject-year. Though not required for merging in R\footnote{Unlike in other statistical programs.} some statistical analyses assume that the data is ordered in a specific way. Well ordered data is also easier for people to read.

We can order observations in our data set using the \texttt{order} command.\index{R command!order}\index{sort}\index{order} For example, to order \emph{MoltenFert} by country-year we type:

\begin{knitrout}
\definecolor{shadecolor}{rgb}{0.969, 0.969, 0.969}\color{fgcolor}\begin{kframe}
\begin{alltt}
\hlcomment{# Order MoltenFert by country-year}
MoltenFert <- MoltenFert[\hlfunctioncall{order}(MoltenFert$country,
                                MoltenFert$year), ]
\end{alltt}


{\ttfamily\noindent\bfseries\color{errorcolor}{\#\# Error: argument 2 is not a vector}}\begin{alltt}

\hlcomment{# Show MoltenFert}
\hlfunctioncall{head}(MoltenFert)
\end{alltt}
\begin{verbatim}
##   iso2c        country variable  value
## 1    AF    Afghanistan     2002  3.403
## 2    AL        Albania     2002 97.185
## 3    DZ        Algeria     2002  9.642
## 4    AS American Samoa     2002     NA
## 5    AD        Andorra     2002     NA
## 6    AO         Angola     2002  1.659
\end{verbatim}
\end{kframe}
\end{knitrout}


\subsection{Subsetting data}\index{R!subsetting data}

Sometimes you may want to use only a subset of a data frame. For example, the density plot in Figure \ref{FertilizerConsumptionDens} shows us that the \emph{MoltenFert} data has a few very extreme values. We can use the \texttt{subset}\index{subset}\index{R command!subset} command to examine these outliers,\index{outliers} for example countries that have fertilizer consumption greater than 1000 kilograms per hectare.  

\begin{figure}
    \caption{Density Plot of Fertilizer Consumption (kilograms per hectare of arable land)}
    \label{FertilizerConsumptionDens}






































