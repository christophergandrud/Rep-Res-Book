%%%%%%%%%%%%%%%
% Parent document for the book Reproducible Research with R and RStudio
% Christopher Gandrud
% 16 January 2013
%%%%%%%%%%%%%%

% Tell RStudio that weaving is to be done with the knitr package
% !Rnw weave = knitr

\documentclass[krantz1]{krantz}

% Load required LaTeX packages
\usepackage[authoryear]{natbib}
\usepackage{amssymb}
\usepackage{amsmath}
\usepackage{graphicx}
%\usepackage{caption}
\usepackage{subfigure}
%\usepackage{epsfig}
\usepackage{makeidx}
%\usepackage{showidx}
\usepackage{multicol}
	\frenchspacing
	\tolerance=5000
\usepackage[usenames,dvipsnames,svgnames]{xcolor}
\usepackage{dcolumn}
\usepackage{booktabs}
\usepackage{multirow}
\usepackage{lscape}
\usepackage{url}
\usepackage{todonotes}
\usepackage{tikz}
	\usetikzlibrary{trees}
	\usetikzlibrary{decorations.pathmorphing}
	\usetikzlibrary{shapes,arrows}
\usepackage{wrapfig}
\usepackage{alltt}

\newcommand{\blankpage}{
	\newpage
	\thispagestyle{empty}
	\mbox{}
	\newpage
}


% Set margins on highlighted code output boxes
\setlength\fboxsep{6.25mm}

% Set knitr global options



\makeatletter
\makeatother
\makeindex

\begin{document}

% Title page
\title{Reproducible Research with R and RStudio}

\author{Christopher Gandrud}

\maketitle

% Set roman numeral page counter
\pagenumbering{roman}

% Front matter
\frontmatter

\blankpage
\blankpage
\blankpage
\blankpage

% Preface



\chapter*{Preface}

\noindent This book has its genesis in my PhD research as the London School of Economics. I started it with questions about the 2008/09 financial crisis and planned to spend most of my time researching about capital adequacy requirements. But I quickly realized much of my time would be spent learning the day-to-day tasks of carrying out and tying together data gathering, analysis, and results presentation. After plodding through of awhile, the breaking point came while reentering results in to a regression table after I had tweaked one of my statistical models, yet again. Surely there was a better way that would allow me to spend more time answering my research questions and less time typing similar things over and over. Making your research reproducible for others also means making it better organized and efficient for yourself. So, my search for a better way led me straight to the tools for reproducible computational research. 

The reproducible research community is very active, knowledgable and helpful. Nonetheless, I often encountered holes in this collective knowledge, or at least no resource to bring it all together as a whole. That is my intention for this book: to bring together the skills I have picked up for actually doing and presenting computational research. Hopefully, the book along with making reproducible research more prevalent, will save researchers hours of Googling, so they can spend more time addressing their research questions.  

I would not have been able to write this book without many people's advice and support. Foremost is John Kimmel, acquisitions editor at Chapman \& Hall. He approached me with in Spring 2012 with the general idea and opportunity for this book. The developer and blogging community has been incredibly important for making this book possible. Foremost among these people is Yihui Xie. He is the main developer behind the {\emph{knitr}} package and also an avid writer and commenter of blogs. Without him the ability to do reproducible research would be much harder and the blogging community that spreads knowledge about how to do these things would be poorer. Other great contributors to the reproducible research community include Carl Boettiger, Markus Gesmann (who developed {\emph{googleVis}}), Jeromy Anglim, Rob Hyndman and especially Hadley Wickham (who has developed numerous R packages) and Ramnath Vaidyanathan (who developed {\emph{Slidify}} and is an active {\emph{knitr}} contributor).

The vibrant community at Stack Overflow \url{http://stackoverflow.com/} and Stack Exchange \url{http://stackexchange.com/} are always very helpful for finding answers to problems that plague any computational researcher. Importantly the sites make it easy for others to find the answers to questions that have already been asked. Thank you also to Victoria Stodden, Michael Malecki and a number of anonymous reviewers for helpful suggestions.

My students at Yonsei University were also an important part of creating this book. One of the reasons that I got interested in using many of the tools covered in this book like using \emph{knitr} in slideshows, was to improve a course I taught there: Introduction to Social Science Data Analysis. I tested many of the explanations and examples in this book on my students. Their feedback has been very helpful for making the book clearer and more useful. Their experience with using these tools on Windows computers was also important for improving the book's Windows documentation.

My wife, Kristina Gandrud, has been immensely supportive and patient with me throughout the writing of this book (and pretty much my entire academic career). Certainly this is not the proper forum for musing about marital relations, but I'll add a musing anyways. Having a person who supports your interests, even if they don't completely share them is immensely helpful for a researcher. It keeps you going.


% Table of Contents
\tableofcontents

% Convert Stylistic Conventions child documnet from Markdown to LaTex and include
\chapter*{Stylistic Conventions}\label{StylisticConventions}




I use the following conventions throughout this book to format computer
code and actions:

\begin{itemize}
\item
  Abstract Variables
\end{itemize}
Abstract variables, i.e.~variables that do not represent specific
objects in an example, are in \texttt{ALL CAPS TYPWRITER TEXT}.

\begin{itemize}
\item
  Clickable Buttons
\end{itemize}
Clickable Buttons are in \texttt{typewriter text}.

\begin{itemize}
\item
  Code
\end{itemize}
All code is in \texttt{typewriter text}.

\begin{itemize}
\item
  Filenames and Directories
\end{itemize}
Filenames and directories more generally are printed in \emph{italics}.

Camelback is used for file and directory names.

\begin{itemize}
\item
  Individual variable values
\end{itemize}
Individual variable values mentioned in the text are in \textbf{bold}.

\begin{itemize}
\item
  Objects
\end{itemize}
Objects are printed in \emph{italics}.

Camelback is used for objects.

\begin{itemize}
\item
  Object Classes
\end{itemize}
Object classes are in \texttt{typewriter text}.

\begin{itemize}
\item
  Packages
\end{itemize}
\textbf{R} packages are printed in \emph{italics}.

\begin{itemize}
\item
  Windows
\end{itemize}
Open windows are written in \textbf{bold} text.

\begin{itemize}
\item
  Variable Names
\end{itemize}
Variable names are printed in \emph{italics}.

Camelback (e.g.~CamelBack) is used for individual variable names.


% Include page on installing R packages used in the book



\chapter*{Required R Packages} \label{ReqPackages}

In this book I discuss how to use a number of user-written R packages for reproducible research. Many of these packages are not included in the default R installation. They need to be installed separately. To install all of the user-written packages discussed in this book copy the following code and paste it into your R console:

\begin{knitrout}
\definecolor{shadecolor}{rgb}{0.969, 0.969, 0.969}\color{fgcolor}\begin{kframe}
\begin{alltt}
\hlfunctioncall{install.packages}(\hlstring{"apsrtable"}, \hlstring{"brew"}, \hlstring{"countrycode"},
                \hlstring{"devtools"}, \hlstring{"formatR"}, \hlstring{"gdata"}, \hlstring{"ggplot2"}, 
                \hlstring{"googleVis"}, \hlstring{"httr"}, \hlstring{"knitr"}, 
                \hlstring{"markdown"}, \hlstring{"openair"}, \hlstring{"plyr"}, 
                \hlstring{"quantmod"}, \hlstring{"reshape"}, \hlstring{"reshape2"},
                \hlstring{"RCurl"}, \hlstring{"rjson"}, \hlstring{"RJSONIO"}, \hlstring{"stargazer"},
                \hlstring{"texreg"}, \hlstring{"tools"}, \hlstring{"treebase"},
                \hlstring{"twitteR"}, \hlstring{"WDI"}, \hlstring{"XML"},                 
                \hlstring{"xtable"}, \hlstring{"Zelig"})
\end{alltt}
\end{kframe}
\end{knitrout}


\noindent Once you enter this code, you may be asked to select a CRAN ``mirror"\index{CRAN!mirror} to download the packages from.\footnote{CRAN stands for the Comprehensive R Archive Network.} Simply select the mirror closest to you.

 Ramnath Vaidyanathan's \emph{Slidify}\index{Slidify} package \citeyearpar{R-slidify} for creating R Markdown/HTML slideshows (see Chapter \ref{MarkdownChapter}) is not currently on CRAN. It can be downloaded directly from GitHub\index{GitHub}. To do this first load the \emph{devtools} package \cite[]{R-devtools}. Then download \emph{Slidify}. Here is the complete code:

\begin{knitrout}
\definecolor{shadecolor}{rgb}{0.969, 0.969, 0.969}\color{fgcolor}\begin{kframe}
\begin{alltt}
\hlcomment{# Load devtools}
\hlfunctioncall{library}(devtools)

\hlcomment{# Install Slidify and ancillary libraries}
\hlfunctioncall{install_github}(\hlstring{"slidify"}, \hlstring{"ramnathv"})
\hlfunctioncall{install_github}(\hlstring{"slidifyLibraries"}, \hlstring{"ramnathv"})
\end{alltt}
\end{kframe}
\end{knitrout}


\noindent For more details see the \emph{Slidify} website: \url{http://ramnathv.github.com/slidify/start.html#}.

If you are using Windows you will also need to install \emph{Rtools} \cite[]{Rtools}.\index{Rtools} You can download Rtools from: \url{http://cran.r-project.org/bin/windows/Rtools/}.\label{RtoolsDownload} Please use the recommended installation to ensure that your system PATH\index{PATH} is set up correctly. Otherwise your computer will not know where the tools are.





\listoffigures
\listoftables


\mainmatter

% Start arabic numeral page counter
\setcounter{page}{1}

% Part 1, include child documents
\part{Getting Started}

% Chapter Chapter 1 For Reproducible Research in R and RStudio
% Christopher Gandrud
% Created: 16/07/2012 05:45:03 pm CEST
% Updated: 27 January 2013




\chapter{Introducing Reproducible Research}\label{Intro}

Research is often presented in very abridged packages: slideshows, journal articles, books, or maybe even websites. These presentation documents announce a project's findings and try to convince us that the results are correct \cite[]{Mesirov2010}. It's important to remember that these documents are not the research. Especially in the computational and statistical sciences, these documents are the ``advertising". The research is the ``full software environment, code, and data that produced the results" \cite[385]{Buckheit1995,Donohue2010}. When we separate the research from its advertisement we are making it difficult for others to verify the findings by reproducing them. 

This book gives you the tools to dynamically combine your research with the presentation of your findings. The first tool is a workflow for reproducible research that weaves the principles of reproducibility throughout your entire research project, from data gathering to the statistical analysis, and the presentation of results. You will also learn how to use a number of computer tools that make this workflow possible. These tools include:

\begin{itemize}
    \item the R statistical language that will allow you to gather data and analyze it,
    \item the LaTeX and Markdown markup languages that you can use to create documents--slideshows, articles, books, and webpages--for presenting your findings,
    \item the {\emph{knitr}} package for R and other tools, including command line programs like GNU Make and Git version control, for dynamically tying your data gathering, analysis, and presentation documents together so that they can be easily reproduced,
    \item RStudio, a program that brings all of these tools together in one place.
\end{itemize}

%%%%%%%%%%%%%% What is reproducible research? %%%%%%%%%%%%%

\section{What is reproducible research?}

Research results are replicable if there is sufficient information available for independent researchers to make the same findings using the same procedures \cite[444]{King1995}. For research that relies on experiments, this can mean a researcher not involved in the original research being able to rerun the experiment and validate that the new results match the original ones. In computational and quantitative empirical sciences results are replicable if independent researchers can recreate findings by following the procedures originally used to gather the data and run the computer code. Of course it is sometimes difficult to replicate the original data set because of limited resources.\footnote{In this book we will actually aim for replicable research, even if we don't always achieve it. New technologies make it possible to easily replicate some kinds of data sets, especially if the original data is available over the internet.} So as a next-best standard we can aim for ``really reproducible research" \cite[1226]{Peng2011}.\footnote{The idea of really reproducible computational research was originally thought of and implemented by Jon Claerbout\index{Jon Claerbout} and the Stanford Exploration Project beginning in the 1980s and early 1990s \cite[]{Fomel2009,Donohue2009}. Further seminal advances were made by Jonathan B. Buckheit and David L. Donohue who created the Wavelab library of MatLab\index{MatLab} routines for their research on wavelets in the mid-1990s \cite[]{Buckheit1995}.} In computational sciences\footnote{Reproducibility is important for both quantitative and qualitative research \cite[]{King1994}. Nonetheless, we will focus mainly on on methods for reproducibility in quantitative computational research.} this means:

\begin{quote}
    the data and code used to make a finding are available and they are sufficient for an independent researcher to recreate the finding.
\end{quote} 

In practice, research needs to be {\emph{easy}} for independent researchers to reproduce \cite[]{Ball2012}. If a study is difficult to reproduce it's more likely that no one will reproduce it. If someone does attempt to reproduce this research, it will be difficult for them to tell if any errors they find were in the original research or problems they introduced during the reproduction. In this book you will learn how to avoid these problems. 

In particular you will learn tools for dynamically ``{\emph{knitting}}"\index{knit}\footnote{Much of the reproducible computational research and literate programming literatures have traditionally used the term ``weave"\index{weave} to describe the process of combining source code and presentation documents \cite[see][101]{Knuth1992}. In the R community weave is usually used to describe the combination of source code and LaTeX documents. The term ``knit" reflects the vocabulary of the {\emph{knitr}} R package\index{knitr} (knit + R). It is used more generally to describe weaving with a variety of markup languages. Because of this, I use the term knit rather than weave in this book.} the data and the source code together with your presentation documents. Combined with well organized source files and clearly and completely commented code, independent researchers will be able to understand how you obtained your results. This will make your computational research easily reproducible.

%%%%%%%%%%%%%% Why should research be reproducible? %%%%%%%%%%%%%

\section{Why should research be reproducible?}

Reproducibility research is one of the main components of science. If that's not enough reason for you to make your research reproducible, consider that the tools of reproducible research also have direct benefits for you as a researcher. 

\subsection{For Science}

Replicability has been a key part of scientific enquiry from perhaps the 1200s \cite[]{Bacon1267,Nosek2012}. It has even been called the ``demarcation between science and non-science" \cite[2]{Braude1979}. Why is replication so important for scientific inquiry? 

\paragraph{Standard to judge scientific claims} 
Replication, or at the least reproducibility, opens claims to scrutiny; allowing us to keep what works and discard what doesn't. Science, according to the American Physical Society, ``is the systematic enterprise of gathering knowledge \ldots organizing and condensing that knowledge into testable laws and theories." The ``ultimate standard" for evaluating these scientific claims is whether or not the claims can be replicated \cite[]{Peng2011,Kelly2006}. Research findings cannot even really be considered ``genuine contribution[s] to human knowledge" until they have been verified through replication \cite[38]{Stodden2009}. Replication ``requires the complete and open exchange of data, procedures, and materials". Scientific conclusions that are not replicable should be abandoned or modified ``when confronted with more complete or reliable \ldots evidence".\footnote{See the American Physical Society's website at \url{http://www.aps.org/policy/statements/99_6.cfm}. See also \cite{Fomel2009}.} 

\paragraph{Avoiding effort duplication \& encouraging cumulative knowledge development} 
Not only is reproducibility crucial for evaluating scientific claims, it can also help enable the cumulative growth of future scientific knowledge \citep{Kelly2006,King1995}. Reproducible research cuts down on the amount of time scientists have to spend gathering data or developing procedures that have already been collected or figured out. Because researchers do not have to discover on their own things that have already been done, they can more quickly apply these data and procedures to building on established findings and developing new knowledge.

\subsection{For You}

Working to make your research reproducible does require extra upfront effort. For example, you need to put effort into learning the tools of reproducible research by doing things such as reading this book. But beyond the clear benefits for science, why should you make this effort? Using reproducible research tools can make your research process more effective and (hopefully) ultimately easier.

\paragraph{Better work habits}
Making a project reproducible from the start encourages you to use better work habits. It can spur you to more effectively plan and organize your research. It should push you to bring you data and source code up to a higher level of quality than you might if you ``thought `no one was looking'" \cite[386]{Donohue2010}. This forces you to root out errors--a ubiquitous part of computational research-earlier in the research process \cite[385]{Donohue2010}. Clear documentation also makes it easier to find errors.\footnote{Of course, it's important to keep in mind that reproducibility is ``neither necessary nor sufficient to prevent mistakes" \cite[]{Stodden2009b}.}

Reproducible research needs to be stored so that other researchers can actually access the data and source code. By taking steps to make you research accessible for others you are also making it easier for you to find your data and methods when you revise your work or begin new projects. You are avoiding personal effort duplication; allowing you to cumulatively build on your own work more effectively.

\paragraph{Better teamwork}
The steps you take to make sure an independent researcher can figure out what you have done also make it easier for your collaborators to understand your work and build on it. This applies not only to current collaborators, but also future collaborators. Bringing new members of a research team up to speed on a cumulatively growing research project is faster if they can easily understand what has been done already \cite[386]{Donohue2010}. 

\paragraph{Changes are easier}
A third person may or may not actually reproduce your research even if you make it easy for them to do so. But, {\emph{you will almost certainly reproduce parts or even all of your own research}}. Almost no actual research process is completely linear. You almost never gather data, run analyses, and present you results without going backwards to add variables, make changes to your statistical models, create new graphs, alter results tables in light of new findings, and so on. You will probably try to make these changes long after you last worked on the project and long since you remembered the details of how you did it. Whether your changes are because of journal reviewers' and conference participants' comments or you discover that new and better data has been made available since beginning the project, designing your research to be reproducible from the start makes it much easier to change things later on.  

Dynamically reproducible documents in particular can make changing things much easier. Changes made to one part of a research project have a way of cascading through the other parts. For example, adding a new variable to a largely completed analysis requires gathering new data and merging it with existing data sets. If you used data imputation or matching methods you may need to rerun these models. You then have to update your main statistical analyses, and recreate the tables and graphs you used to present the results. Adding a new variable essentially forces you to reproduce large portions of your research. If when you started the project you used tools that make it easier for others to reproduce your research, you also made it easier to reproduce the work yourself. You will have taken steps to have a ``better relationship with [your] future [self]" \cite[]{Bowers2011}.

\paragraph{Higher research impact}
Reproducible research is more likely to be useful for other researchers than non-reproducible research. Useful research is cited more frequently \cite[]{Donohue2002,Piwowar2007,Vandewalle2012}. Research that is fully reproducible contains more information, i.e. more reasons to use and cite it, than presentation documents merely showing findings. Independent researchers may use the reproducible data or code to look at other, often unanticipated, questions. When they use your work for a new purpose they will (should) cite your work. Because of this, Vandewalle et al. even argue that ``the goal of reproducible research is to have more impact with our research"  \citeyearpar[1253]{Vandewalle2007}.

A reason researchers often avoid making their research fully reproducible is that they are afraid other people will use their data and code to compete with them. I'll let Donohue et al. address this one:

\begin{quote}
    True. But competition means that strangers will read your papers, try to learn from them, cite them, and try to do even better. If you prefer obscurity, why are you publishing? \citeyearpar[16]{Donohue2009}
\end{quote}

\section{Who should read this book?}

This book is intended primarily for researchers who want to use a systematic workflow that encourages reproducibility and the practical state-of-the-art computer tools to put this workflow into practice. This includes professional researchers, upper-level undergraduate, and graduate students working on computational data-driven projects. Hopefully, editors at academic publishers will also find the book useful for improving their ability to evaluate and edit reproducible research. 

The more researchers that use the tools of reproducibility the better. So I include enough information in the book for people who have very limited experience with these tools, including limited experience with R, LaTeX, and Markdown. They will be able to start incorporating these tools into their workflow right away. The book will also be helpful for people who already have general experience using technologies such as R and LaTeX, but would like to know how to tie them together for reproducible research. 

\subsection{Academic Researchers}
Hopefully so far in this chapter I've convinced you that reproducible research has benefits for you as a member of the scientific community and personally as a computational researcher. This book is intended to be a practical guide for how to actually make your research reproducible. Even if you already use tools such as R and LaTeX you may not be leveraging their full potential. This book will teach you useful ways to get the most out of them as part of a reproducible research workflow.

\subsection{Students}
Upper-level undergraduate and graduate students conducting original computational research should make their research reproducible for the same reasons that professional researchers should. Forcing yourself to clearly document the steps you took will also encourage you to think more clearly about what you are doing and reinforce what you are learning. It will hopefully give you a greater appreciation of research accountability and integrity early in your career \cite[183]{Barr2012,Ball2012}.

Even if you don't have extensive experience with computer languages, this book will teach you specific habits and tools that you can use throughout your student research and hopefully your careers. Learning these things earlier will save you considerable time and effort later.

\subsection{Instructors}
When instructors incorporate the tools of reproducible research into their assignments they not only build students' understanding of research best practice, but are also better able to evaluate and provide meaningful feedback on students' work \cite[183]{Ball2012}. This book provides a resource that you can use with students to put reproducibility into practice.

If you are teaching computational courses, you may also benefit from making your lecture material dynamically reproducible. Your slides will be easier to update for the same reasons that it is easier to update research.  Making the methods you used to create the material available to students will give them more information. Clearly documenting how you created lecture material can also pass information on to future instructors. 

\subsection{Editors}
Beyond a lack of reproducible research skills among researchers, an impediment to actually creating reproducible research is a lack of infrastructure to publish it \cite[]{Peng2011}. Hopefully, this book will be useful for editors at academic publishers who want to be better at evaluating reproducible research, editing it, and developing systems to make it more widely available. The journal {\emph{Biostatistics}} is a good example of a publication that is encouraging (actually requiring) reproducible research. From 2009 the journal  has had an editor for reproducibility that ensures replication files are available and that results can be replicated using these files \cite[]{Peng2009}. The more editors there are with the skills to work with reproducible research the more likely it is that researchers will do it.

\subsection{Private sector researchers}

Researchers in the private sector may or may not want to make their work easily reproducible outside of their organization. However, that does not mean that significant benefits cannot be gained from using the methods of reproducible research. First, even if public reproducibility is ruled out to guard proprietary information,\footnote{There are ways to enable some public reproducibility without revealing confidential information. See \cite{Vandewalle2007} for a discussion of one approach.} making your research reproducible to members of your organization can spread valuable information about how analyses were done and data was collected. This will help build your organization's knowledge and avoid effort duplication. Just as a lack of reproducibility hinders the spread of information in the scientific community, it can hinder it inside of a private organization. 

Also, the tools of reproducible research covered in this book enable you to create professional standardized reports that can be easily updated or changed when new information is available. In particular, you will learn how to create batch reports based on quantitative data.

%%%%%%%%%%%%%%%%% The Tools of Reproducible Research %%%%%%%%%%%%%%%

\section{The Tools of Reproducible Research}

This book will teach you the tools you need to make your research highly reproducible. Reproducible research involves two broad sets of tools. The first is a {\bf{reproducible research environment}}\index{reproducible research!environment} that includes the statistical tools you need to run your analyses as well as ``the ability to automatically track the provenance of data, analyses, and results and to package them (or pointers to persistant versions of them) for redistribution". The second set of tools is a {\bf{reproducible research publisher}}\index{reproducible research!publisher}, which prepares dynamic documents for presenting results and is easily linked to the reproducible research environment \cite[415]{Mesirov2010}.

In this book we will focus on learning how to use the widely available and highly flexible reproducible research environment--R/RStudio \cite[]{RLanguage,RStudioCite}. R/RStudio can be linked to numerous reproducible research publishers such as LaTeX and Markdown with Yihui Xie's {\emph{knitr}} package \citeyearpar{R-knitr}. The main tools covered in this book include:

\begin{itemize}
    \item {\bf{R}}: a programming language primarily for statistics and graphics. It can also be useful for data gathering and creating presentation documents.
    
    \item {\bf{{\emph{knitr}}}}: an R package for literate programming\index{literate programming}, i.e. it allows you to combine your statistical analysis and the presentation of the results into one document. It works with R and a number of other languages such as Bash, Python, and Ruby.
    
    \item {\bf{Markup languages}}: instructions for how to format a presentation document. In this book we cover LaTeX, Markdown, and a little HTML.  
    
    \item {\bf{RStudio}}: an integrated developer environment (IDE)\index{integrated developer environment} for R that tightly integrates R, {\emph{knitr}}, and markup languages.
    
    \item {\bf{Cloud storage \& versioning}}: Services such as Dropbox and Github that can store data, code, and presentation files, save previous versions of these files, and make this information widely available.
    
    \item {\bf{Unix-like shell programs}}\index{Unix-like shell program}: These tools are useful for working with large research projects.\footnote{In this book I cover the Bash shell for Linux\index{Linux} and Mac as well as Windows PowerShell\index{Windows PowerShell}.} They also allow us to use command line tools including GNU Make for compiling projects and Pandoc, a program useful for converting documents from one markup language to another.
\end{itemize}

%%%%%%%%%%%%%%%%%%% Why use R, knitr, and RStudio for reproducible research? %%%%%%%%%%%%%%

\section{Why use R, knitr, and RStudio for reproducible research?}

\paragraph{Why R?}
Why use a statistical programming language like R for reproducible research? R has a very active development community that is constantly expanding what it is capable of. As we will see in this book this enables researchers across a wide range of disciplines to gather data and run statistical analyses. Using the {\emph{knitr}} package, you can connect your R-based analysess to presentation documents created with markup languages\index{markup language} such as LaTeX and Markdown. This allows you to dynamically and reproducibly present results in articles, slideshows, and webpages. 

The way you interact with R has benefits for reproducible research. In general you interact with R (or any other programming and markup language) by explicitly writing down your steps as source code\index{source code}. This promotes reproducibility more than your typical interactions with Graphical User Interface (GUI)\index{Graphical User Interface}\index{GUI} programs like\index{SPSS} SPSS\footnote{I know you can write scripts in statistical programs like SPSS, but doing so is not encouraged by the program's interface and you often have to learn multiple languages for writing scripts that run analyses, create graphics, and deal with matrices.} and Microsoft Word\index{Microsoft Word}. When you write R code and embed it in presentation documents created using markup languages you are forced to explicitly state the steps you took to do your research. When you do research by clicking through drop down menus in GUI programs, your steps are lost, or at least documenting them requires considerable extra effort. Also it is generally more difficult to dynamically embed your analysis in presentation documents created by GUI word processing programs in a way that will be accessible to other researchers both now and in the future. I'll come back to these points in Chapter \ref{GettingStartedRR}.

\paragraph{Why knitr?}

Literate programming\index{literate programming} is a crucial part of reproducible quantitative research.\footnote{Donald Knuth\index{Donald Knuth} coined the term literate programming in the 1970s to refer to a source file that could be both run by a computer and ``woven" with a formatted presentation document \cite[]{Knuth1992}.} Being able to directly link your analyses, your results, and the code you used to produce the results makes tracing your steps much easier. There are many different literate programming tools for a number of different programming languages. Previously, one of the most common tools for researchers using R and the LaTeX markup language was Sweave \cite[]{Leisch2002}.\index{Sweave} The package I am going to focus on in this book is newer and is called {\emph{knitr}}\index{knitr}. Why are we going to use {\emph{knitr}} in this book and not Sweave or some other tool?

The simple answer is that {\emph{knitr}} has the same capabilities as Sweave and more. It can work with markup languages other than LaTeX\footnote{It works with LaTeX, Markdown, HTML, and reStructuredText\index{reStructuredText}. We cover the first two in this book.} and can even work with programming languages other than R. It highlights R code\index{syntax highlighting} in presentation documents making it easier for your readers to follow.\footnote{Syntax highlighting uses different colors and fonts to distinguish different types of text. For example in the PDF version of this book R commands are highlighted in \hlfunctioncall{maroon}, while character strings are in \hlstring{lavender}.} It gives you better control over the inclusion of graphics and can cache code chunks--save the output for later.\index{cache} It has the ability to understand Sweave-like syntax, so it will be easy to convert backwards to Sweave if you want to.\footnote{Note that the Sweave-style syntax is not identical to actual Sweave syntax. See Yihui Xie's discussion of the differences between the two at: \url{http://yihui.name/knitr/demo/sweave/}. \emph{knitr} has a function (\texttt{Sweave2knitr})\index{Sweave2knitr} for converting Sweave to \emph{knitr} syntax.} You also have the choice to use much simpler and more straightforward syntax with {\emph{knitr}}. 

\paragraph{Why RStudio?}

\index{RStudio}Why use the RStudio integrated development environment for reproducible research? R by itself has the capabilities necessary to gather data, analyse it, and, with a little help from {\emph{knitr}} and markup languages, present results in a way that is highly reproducible. RStudio allows you to do all of these things, but simplifies many of them and allows you to navigate through them more easily. It is a happy medium between R's text-based interface and a pure GUI. 

Not only does RStudio do many of the things that R can do, but more easily, it is also a very good stand alone editor for writing documents with LaTeX and Markdown. For LaTeX documents it can, for example, insert frequently used commands like \texttt{\textbackslash{}section\{\}} for numbered sections (see Chapter \ref{LatexChapter}).\footnote{If you are more comfortable with a what-you-see-is-what-you-get (WYSIWYG)\index{WYSIWYG} word processor like Microsoft Word, you might be interested in exploring Lyx\index{Lyx}. It is a WYSIWYG-like LaTeX editor that works with {\emph{knitr}}. It doesn't work with the other markup languages covered in this book. For more information see: \url{http://www.lyx.org/}. I give some brief information on using Lyx with \emph{knitr} in Chapter 3's Appendix.}  There are many LaTeX editors available, both open source and paid. But RStudio is currently the best program for creating reproducible LaTeX and Markdown documents. It has full syntax highlighting\index{syntax highlighting}. It's syntax highlighting can even distinguish between R code and markup commands in the same document. It can spell check LaTeX and Markdown documents. It handles {\emph{knitr}} code chunks\index{code chunk} beautifully (see Chapter \ref{GettingStartedRKnitr}). 

Finally, RStudio not only has tight integration with various markup languages, it also has capabilities for using other tools such as C++, CSS, JavaScript, and a few other programming languages. It is closely integrated with the version control programs Git\index{Git} and SVN\index{SVN}. Both of these programs allow you to keep track of the changes you make to your documents (see Chapter \ref{Storing}). This is important for reproducible research since version control programs can document many of your research steps. Basically, RStudio makes it easy to create and navigate through complex reproducible research documents.  

\subsection{Installing the main software}\label{InstallR}

Before you read this book you should install the main software. All of the software programs covered in this book are open source and can be easily downloaded for free. They are available for Windows\index{Windows}, Mac\index{Mac}, and Linux operating systems\index{Linux}. They should run well on most modern computers. 

You should install R before installing RStudio. You can download the programs from the following websites:

\begin{itemize}
    \item {\bf{R}}: \url{http://www.r-project.org/},
    \item {\bf{RStudio}}: \url{http://www.rstudio.com/ide/download/}.
\end{itemize}

\noindent The download webpages for these programs have comprehensive information on how to install them, so please refer to those pages for more information.

After installing R and RStudio you will probably also want to install a number of user-written packages that are covered in this book. To install all of these user-written packages, please see page \pageref{ReqPackages}.

\paragraph{Installing markup languages}\label{InstallMarkup}

If you are planning to create LaTeX documents you need to install a TeX distribution\index{TeX distribution}.\footnote{LaTeX is is really a set of macros for the the TeX typesetting system.\index{TeX} It is included in all major TeX distributions.} They are available for Windows, Mac, and Linux systems. They can be found at: \url{http://www.latex-project.org/ftp.html}. Please refer to that site for more installation information.

If you want to create Markdown documents you can separately install the {\emph{markdown}} package\index{R package!markdown}} in R. You can do this the same way that you install any package in R, with the {\tt{install.packages}} command.\footnote{The exact command is: {\tt{install.packages("markdown")}}.} 

\paragraph{GNU Make}

If you are using a Linux computer you already have GNU Make\label{InstallMake}\index{GNU Make} installed.\footnote{To verify this open the Terminal\index{Terminal} and type: \texttt{make --version. I used version 3.81 for this book.}. This should output details about the current version of Make installed on your computer.} Mac users should go to the App Store\index{Apple App Store} and download Xcode (it's free).\index{Xcode} Once Xcode is installed, install command line tools, which you will find by opening Xcode clicking on \texttt{Preference} \textrightarrow \texttt{Downloads}. Windows users will have Make installed if they have already installed Rtools\index{Rtools} (see page \pageref{RtoolsDownload}). Mac and Windows users will need to install this software not only so that GNU Make runs properly, but also so that other command line tools work well.

\paragraph{Other Tools}

We will discuss other tools such as Git that can be a useful part of a reproducible research workflow. Installation instructions for these tools will be discussed below.

%%%%%%%%%%%%%% Book Overview %%%%%%%%%%%%%%

\section{Book overview}

The purpose of this book is to give you the tools that you will need to do reproducible research with R and RStudio. This book describes a workflow for reproducible research primarily using R and RStudio. It is designed to give you the necessary tools to use this workflow for your own research. It is not designed to be a complete reference for R, RStudio, {\emph{knitr}}, Git, or any other program that is a part of this workflow. Instead it shows you how these tools can fit together to make your research more reproducible. To get the most out of these individual programs I will along the way point you to other resources that cover these programs in more detail.

To that end, I can recommend a number of resources that cover more of the nitty-gritty:

\begin{itemize}
    \item Michael J. Crawley's \citeyearpar{Crawley2013} encyclopaedic R book, appropriately titled, \textbf{The R Book} published by Wiley.
    
    \item Similarly, Robert I. Kabacoff's \citeyearpar{Kabacoff2012} useful book \textbf{R in Action} published by Manning. He also maintains a very helpful website called Quick-R (\url{http://www.statmethods.net/}).

    \item Norman Matloff's \citeyearpar{Matloff2011} tour through the programming language aspects of  R called \textbf{The Art of R Programming: A Tour of Statistical Design Software} published by No Starch Press.
    
    \item For an excellent introduction to the command line\index{command line} in Linux and Mac, though also helpful for Windows users running PowerShell (see Chapter 2) see William E. Shotts Jr.'s \citeyearpar{ShottsJr2012} book \textbf{The Linux Command Line: A Complete Introduction} also published by No Starch Press.
    
    \item The RStudio website (\url{http://www.rstudio.com/ide/docs/}) has a number of useful tutorials on how to use {\emph{knitr}} with LaTeX and Markdown.
\end{itemize}

That being said, my goal is for this book to be {\emph{self-sufficient}}. A reader without a detailed understanding of these programs will be able to understand and use the commands and procedures I cover in this book. While learning how to use R and the other programs I personally often encountered illustrative examples that included commands, variables, and other things that were not well explained in the texts that I was reading. This caused me to waste many hours trying to figure out, for example, what the \texttt{\$} is used for (preview: it's the component selector). I hope to save you from this wasted time by either providing a brief explanation of possibly frustrating and mysterious things and/or pointing you in the direction of good explanations.

\subsection{How to read this book}

This book gives you a workflow. It has a beginning, middle, and end. So, unlike a reference book it can and should be read linearly as it takes you through an empirical research processes from an empty folder to a completed set of documents that reproducibly showcase your findings.

That being said, readers with more experience using tools like R or LaTeX may want to skip over the nitty-gritty parts of the book that describe how to manipulate data frames or compile LaTeX documents into PDFs. Please feel free to skip these sections.

If you are experienced with R in particular you may want to skip over the first section of Chapter \ref{GettingStartedRKnitr}: Getting Started with R/RStudio. But don't skip over the whole chapter. The latter parts contain important information on the {\emph{knitr}} package. 

Likewise, if you are familiar with LaTeX you might want to skip the first part of Chapter \ref{LatexChapter}. The second part may be useful as it includes information on how to dynamically create BibTeX bibliographies with \emph{knitr} and how to include \emph{knitr} output in a Beamer slideshow.

Conversely, if you do not have experience with LaTeX or Markdown you may benefit from reading, or at least skimming the introductory chapters on these top topics (chapters \ref{LatexChapter} and \ref{MarkdownChapter}) before reading Part III.

When possible I try to give concrete examples in this book that can be used as you read along. A number of longer examples are also accessible on an accompanying GitHub repository (see Chapter \ref{Storing}). You can find it here: \url{https://github.com/christophergandrud/Rep-Res-Examples}.

\subsection{Reproduce this book}

This book practices what it preaches. It can be reproduced. I wrote the book using the programs and methods that I describe. Full documentation and source files can be found at the book's GitHub\index{GitHub} repository. Feel free to read and even use (within reason and with attribution, of course) the book's source code. You can find it at: \url{https://github.com/christophergandrud/Rep-Res-Book}. This is especially useful if you want to know how to do something in the book that I don't directly cover in the text.

\todo[inline]{Before publication, the repository is private and cannot be accessed publicly.} 

\subsection{Contents overview}

The book is broken into four parts. The first part (chapters \ref{GettingStartedRR},  \ref{GettingStartedRKnitr}, and \ref{DirectoriesChapter}) gives an overview of the reproducible research workflow as well as the general computer skills that you'll need to use this workflow. Each of the next three parts of the book guide you through the specific skills you will need for each part of the reproducible research process. Part two (chapters \ref{Storing}, \ref{DataGather}, and \ref{DataClean}) covers the data gathering and file storage process. The third part (chapters \ref{StatsModel}, \ref{TablesChapter}, and \ref{FiguresChapter}) teaches you how to dynamically incorporate your statistical analysis, results figures and tables into your presentation documents. The final part (chapters \ref{LatexChapter}, \ref{LargeDocs}, and \ref{MarkdownChapter}) covers how to create reproducible presentation documents including LaTeX articles, books, slideshows and batch reports as well as Markdown webpages and slideshows.

% Chapter Chapter 2 For Reproducible Research in R and RStudio
% Christopher Gandrud
% Created: 16/07/2012 05:45:03 pm CEST
% Updated: 27 January 2013




\chapter{Getting Started with Reproducible Research}\label{GettingStartedRR}

Researchers often start thinking about making their work reproducible near the end of the research process when they write up their results or maybe even later when a journal requires their data and code be made available for publication. Or maybe even later when another researcher asks if they can use the data from a published article to reproduce the findings. By then there may be numerous versions of the data set and records of the analyses stored across multiple folders on the researcher's computers. It can be difficult and time consuming to sift through these files to create an accurate account of how the results were reached. Waiting until near the end of the research process to start thinking about reproducibility can lead to incomplete documentation that does not give an accurate account of how findings were made. Focusing on reproducibility from the beginning of the process and continuing to follow a few simple guidelines throughout your research can help you avoid these problems. Remember ``reproducibility is not an afterthought--it is something that must be built into the project from the beginning"\cite[386]{Donohue2010}.

This chapter first gives you a brief overview of the reproducible research process: a workflow for reproducible research. Then it covers some of the key guidelines that can help make your research more reproducible.

\section{The Big Picture: A workflow for reproducible research}

The three basic stages of a typical computational empirical research project are:

\begin{itemize}
    \item data gathering,
    \item data analysis,
    \item results presentation.
\end{itemize}

\noindent Each stage is part of the reproducible research workflow covered in this book. Tools for reproducibly gathering data are covered in Part II. Part III teaches tools for tying the data we gathered to our statistical analyses and presenting the results with tables and figures. Part IV discusses how to tie these findings into a variety of documents you can use to advertise your findings.

Instead of starting to use the individual tools of reproducible research  as soon as you learn them I recommend briefly stepping back and considering how the stages of reproducible research {\emph{tie}} together overall. This will make your workflow more coherent from the beginning and save you a lot of backtracking later on.  Figure \ref{WorkflowTies} illustrates the workflow. Notice that the arrows connecting the workflow's parts point in both directions, indicating that you should always be thinking how to make it easier to go backwards through your research, i.e. reproduce it, as well as forwards. 

Around the edges of the figure are some of the commands you will learn to make it easier to go forwards and backwards through the process. These commands tie your research together. For example, you can use API-based R packages\index{API} to gather data from the internet. You can use R's \texttt{merge} command to combine data gathered from different sources into one data set. The \texttt{getURL}\index{R command!getURL} from R's \emph{RCurl} package \citep{R-Rcurl} and the {\texttt{R command!read.table}}\index{read.table} commands can be used to bring this data set into your statistical analyses. The {\emph{knitr}} package then ties your analyses into your presentation documents. This includes the code you used, the figures you created, and, with the help of tools such as the {\emph{xtable}} package, tables of results. You can even tie multiple presentation documents together. For example, you can access the same figure for use in a LaTeX article and a Markdown created website with the \texttt{includegraphics}\index{LaTeX command!includegraphics} and \texttt{![]()}\index{![]()} commands, respectively. This helps you maintain a consistent presentation of results across multiple documents types. We'll cover these commands in detail throughout the book. See Table \ref{TableTieCommands} for a brief, but more complete overview of the main {\emph{tie commands}}.\index{tie commands} 

\subsection{Reproducible Theory}

An important part of the research process that I do not discuss in this book is the theoretical stage. Ideally, if you are using a deductive research design, the bulk of this work will precede and guide the data gathering and analysis stages. Just because I don't cover this stage of the research process doesn't mean that theory building can't and shouldn't be reproducible. It can in fact be ``the easiest part to make reproducible" \cite[1254]{Vandewalle2007}. Quotes and paraphrases from previous works in the literature obviously need to be fully cited so that others can verify that they accurately reflect the source material. For mathematically based theory, clear and complete descriptions of the proofs should be given. 

Though I don't actively cover theory replication in depth in this book, I do touch on some of the ways to incorporate proofs and citations into your presentation documents. These tools are covered in Part IV.

\begin{landscape}
        \begin{figure}
            \caption{Example Workflow \& Commands to Tie it Together}
            \label{WorkflowTies}
                %%%%%%%%%%%%%%
% Reproducible research workflow + link commands
% Christopher Gandrud
% Updated 1 May 2013
%%%%%%%%%%%%%%

% Define colors for figure
%% Color palette (GnBU) chosen using ColorBrewer 2.0
%% See: http://colorbrewer2.org/
%% Not used in the print version
\definecolor{Blue}{HTML}{7BCCC4}
\definecolor{LiteBlue}{HTML}{A8DDB5}
\definecolor{DarkBlue}{HTML}{08589E}

\definecolor{GrayLine}{HTML}{BDBDBD}

% Set node styles
%% Workflow stage nodes
\tikzstyle{Stage} = [draw=Blue,
                     %fill=Blue,
                     rectangle,
                     text width=7em,
                     inner sep=0.5cm,
                     font=\small]

% Raw Data nodes
\tikzstyle{RawData} = [draw=LiteBlue,
                       %fill=LiteBlue,
                       decorate,
                       decoration={random steps,
                                   segment length=2pt,
                                   amplitude=2pt},
                       inner sep=0.25cm,
                       font=\scriptsize]

% Separator line style
\tikzstyle{sepline} = [draw,
                        very thick,
                        color=GrayLine]

% Link command nodes
\tikzstyle{Links} = [draw=none,
                          text width=6em,
                          text=DarkBlue,
                          font=\footnotesize]

% Begin tikz picture
\begin{tikzpicture}

    % Raw Data Nodes
    \node (Data1) at (-3, 7) [RawData]{Raw Data};
    \node (Data2) at (-3, 5) [RawData]{Raw Data};
    \node (Data3) at (-3, 3) [RawData]{Raw Data};

    % Workflow stage nodes
    \node (DataGather) at (0.5, 5) [Stage, text width= 6em]{Data Gather};
    \node (Analysis) at (5.5, 5) [Stage, text width= 4em]{Analysis};
    \node (Presentation1) at (9, 8) [Stage]{LaTeX Book, \\ Article, \& \\ Slideshow \\ Presentations};
    \node (Presentation2) at (9, 2.5) [Stage]{Markdown/ \\ HTML Website \\ Presentations};

    % Lines
    \draw [->, very thick] (Data1) -- (DataGather);
    \draw [->, very thick] (Data2) -- (DataGather);
    \draw [->, very thick] (Data3) -- (DataGather);
    \draw [<->, very thick] (DataGather) -- (Analysis);
    \draw [<->, very thick] (Analysis) -- (Presentation1);
    \draw [<->, very thick] (Analysis) -- (Presentation2);

    \draw [<->, very thick] (Presentation1) -- (Presentation2);

    \path [sepline] (-3.5, 0.75) -- (11, 0.75);
    \path [sepline] (11.5, 9) -- (11.5, 1.5);

    % Link command nodes

    \node (pres) at (13, 5) [Links]{{\emph{knitr}} \\ \texttt{input} \\ \texttt{include} \\ \texttt{includegraphics} \\ Pandoc \\ \texttt{![]()}};
    \node (knitr) at (7.5, -1) [Links]{ {\emph{knitr}} \\ \emph{rmarkdown} \\ \texttt{source} \\ \texttt{source\_url} \\ \texttt{kable} \\ \texttt{print(xtable())} \\ \texttt{texreg} };
    \node (readData) at (3, -1) [Links]{\texttt{source\_data} \\ \texttt{source\_DropboxData} \\ \texttt{read.table} \\ \texttt{getURL} };

    \node (importData) at (-1, -1.3) [Links]{ \texttt{Make} \\ \texttt{download.file} \\ \texttt{source\_data} \\ \texttt{source\_DropboxData} \\ \texttt{read.table} \\ \texttt{merge}\\ \texttt{getURL} \\ API-based packages };


\end{tikzpicture}

        \end{figure}
\end{landscape}

\section{Practical tips for reproducible research}

Before we start learning the details of the reproducible research workflow with R and RStudio it is useful to cover a few broad tips that will help you organize your research process and put these skills in perspective. The tips are:

\begin{enumerate}
    \item Document everything!,
    \item Everything is a (text) file,
    \item All files should be human readable,
    \item Explicitly tie your files together,
    \item Have a plan to organize, store, and make your files available.    
\end{enumerate}

\noindent Using these tips will help make your computational research really reproducible.

\subsection{Document everything!}

In order to reproduce your research others must be able to know what you did. You have to tell them what you did by documenting as much of your research process as possible. Ideally, you should tell your readers how you gathered your data, analyzed it, and presented the results. Documenting everthing is the key to reproducible research and lies behind all of the other tips in this chapter and tools you will learn throughout the book.

\paragraph{Document your R session info}\label{SessionInfoHow}

Before discussing the other tips it's important to learn a key part of documenting with R. You should \emph{record your session info\index{session info}}. Many things in R have stayed the same since it was introduced in the early 1990s. This makes it easy for future researchers to recreate what was done in the past. However, things can change from one version of R to another and from one version of an R package to another. Also, the way R functions and especially how R packages are handled can vary across different operating systems, so it's important to note what system you used. Finally, you may have R set to load packages\index{packages} by default (see page \pageref{Packages} for information about packages). These packages might be necessary to run your code, but other people might not know what packages and what versions of the packages were loaded from just looking at your source code. The \texttt{sessionInfo} command\index{R command!sessionInfo} in R prints a record of all of these things. The information from the session I used to create this book is:

{\footnotesize{
\begin{knitrout}
\definecolor{shadecolor}{rgb}{0.969, 0.969, 0.969}\color{fgcolor}\begin{kframe}
\begin{alltt}
\hlfunctioncall{sessionInfo}()
\end{alltt}
\begin{verbatim}
## R version 2.15.2 (2012-10-26)
## Platform: x86_64-apple-darwin9.8.0/x86_64 (64-bit)
## 
## locale:
## [1] en_GB.UTF-8/en_GB.UTF-8/en_GB.UTF-8/C/en_GB.UTF-8/en_GB.UTF-8
## 
## attached base packages:
##  [1] tools     grid      splines   stats     graphics  grDevices utils    
##  [8] datasets  methods   base     
## 
## other attached packages:
##  [1] Zelig_4.1-1      sandwich_2.2-9   boot_1.3-7       xtable_1.7-0    
##  [5] WDI_2.2          treebase_0.0-6   ape_3.0-7        texreg_1.17     
##  [9] tables_0.7       slidify_0.3.3    stargazer_2.0.1  RCurl_1.95-3    
## [13] bitops_1.0-4.2   reshape2_1.2.2   reshape_0.8.4    quantmod_0.4-0  
## [17] TTR_0.21-1       xts_0.9-3        zoo_1.7-9        Defaults_1.1-1  
## [21] plyr_1.8         openair_0.7-0    memisc_0.95-39   MASS_7.3-23     
## [25] lattice_0.20-13  markdown_0.5.3   Hmisc_3.10-1     survival_2.37-2 
## [29] httr_0.2         googleVis_0.3.3  RJSONIO_1.0-1    ggplot2_0.9.3   
## [33] gdata_2.12.0     formatR_0.7      extrafont_0.13   estout_1.0.1-1  
## [37] devtools_0.8     countrycode_0.10 brew_1.0-6       apsrtable_0.8-8 
## [41] animation_2.1    knitr_1.0.5     
## 
## loaded via a namespace (and not attached):
##  [1] car_2.0-15         cluster_1.14.3     colorspace_1.2-0  
##  [4] data.table_1.8.6   dichromat_2.0-0    digest_0.6.1      
##  [7] evaluate_0.4.3     gee_4.13-18        gtable_0.1.2      
## [10] gtools_2.7.0       labeling_0.1       Matrix_1.0-10     
## [13] memoise_0.1        mgcv_1.7-22        munsell_0.4       
## [16] nlme_3.1-107       parallel_2.15.2    proto_0.3-10      
## [19] RColorBrewer_1.0-5 rjson_0.2.12       Rttf2pt1_1.1      
## [22] scales_0.2.3       stringr_0.6.2      twitteR_0.99.19   
## [25] whisker_0.3-2      XML_3.95-0.1       yaml_2.1.6
\end{verbatim}
\end{kframe}
\end{knitrout}

}}
\noindent Chapter \ref{DirectoriesChapter} gives specific details about how to create files with dynamically included session information.

If you use non-R tools such as Pandoc,\index{Pandoc} you should also record what versions you used.

\subsection{Everything is a (text) file}

Your documentation is stored in files that include data, analysis code, the write up of results, and explanations of these files (e.g. data set codebooks, session info files, and so on). Ideally, you should use the simplest file format possible to store this information. Usually the simplest file format  is the humble, but versatile, text file.\footnote{Plain text files are usually given the file extension \texttt{.txt}. Depending on the size of your data set it may not be feasible to store it as a text file. Nonetheless, text files can still be used for analysis code and presentation files.}

Text files are extremely nimble. They can hold your data in, for example, comma-separated values ({\tt{.csv}}) \index{comma-separated values} format. They can contain your analysis code in {\tt{.R}} files. And they can be the basis for your presentations as markup documents like {\tt{.tex}} or {\tt{.md}}, for LaTeX and Markdown files respectively. All of these files can be opened by any program that can read text files. 

One reason reproducible research is best stored in text files is that this helps {\emph{future proof}} your research. Other file formats, like those used by Microsoft Word \index{Microsoft Word} (\texttt{.docx}) or Excel\index{Microsoft Excel} (\texttt{.xlsx}) change regularly and may not be compatible with future versions of these programs. Text files, on the other hand, can be opened by a very wide range of currently existing programs and, more likely than not, future ones as well. Even if future researchers do not have R or a LaTeX distribution, they will still be able to open your text files and, aided by frequent comments (see below), be able to understand how we conducted your research \cite[3]{Bowers2011}.

Text files are also very easy to search and manipulate with a wide range of programs--such as R and RStudio--that can find and replace text characters as well as merge and separate files. Finally, text files are easy to version and changes can be tracked using programs such as Git (see Chapter \ref{Storing}).   

\subsection{All files should be human readable}

Treat all of your research files as if someone who has not worked on the project will, in the future, try to understand them. Computer code is a way of communicating with the computer. It is `machine readable' in that the computer is able to use it to understand what you want to do.\footnote{Of course, if it does not understand it will usually give us an error message.} However, there is a very good chance that other people (or you six months in the future) will not understand what you were telling the computer. So, you need to make all of your files `human readable'. To make your source code files accessible to other people you need to {\emph{comment frequently}} \cite[3]{Bowers2011} and {\emph{format your code using a style guide}} \cite[]{Nagler1995}. For especially important pieces of code you should use {\emph{literate programming}}--where the source code and the presentation text appear in the same document. Doing this will make it very clear to others how you accomplished a piece of research.

\paragraph{Commenting}
In R everything on a line after a {\tt{\#}} hash character (also known as number, pound, or sharp) is ignored by R, but is readable to people who open the file. The hash character is a comment declaration\index{comment declaration} character. You can use the {\tt{\#}} to place comments telling other people what you are doing. Here are some examples:

\begin{knitrout}
\definecolor{shadecolor}{rgb}{0.969, 0.969, 0.969}\color{fgcolor}\begin{kframe}
\begin{alltt}
\hlcomment{# A complete comment line}
2 + 2  \hlcomment{# A comment after R code}
\end{alltt}
\begin{verbatim}
## [1] 4
\end{verbatim}
\end{kframe}
\end{knitrout}


\noindent On the first line the {\tt{\#}} is placed at the very beginning, so the entire line is treated as a comment. On the second line the {\tt{\#}} is placed after the simple equation \texttt{2 + 2}. R runs the equation as usual and finds the answer {\tt{4}}, but it ignores all of the words after the hash. 

Different languages have different comment declaration characters. In LaTeX everything after the {\tt{\%}} percent sign is treated as a comment and in markdown/HTML comments are placed inside of {\tt{\textless !-- --\textgreater}}. The hash character is used for comment declaration in shell scripts.

Nagler \citeyearpar[491]{Nagler1995} gives some advice on when and how to use comments:

\begin{itemize}
    \item write a comment before a block of code describing what the code does,
    \item comment on any line of code that is ambiguous.
\end{itemize}

\noindent In this book I follow these guidelines when displaying written code. 

He also suggests that all of your source code files should begin with a comment header. {\emph{At the least}} the header should include:

\begin{itemize}
    \item a description of what the file does,
    \item the date it was last updated,
    \item the name of the file's creator and any contributors.
\end{itemize}

\noindent You may also want to include other information in the header such as what other files it depends on, what output files it produces, what version of the programming language you are using or sources that may have strongly influenced the code. 

Here is an example of a minimal file header for an R source code file that creates the third figure in an article titled ``My Article":

\begin{knitrout}
\definecolor{shadecolor}{rgb}{0.969, 0.969, 0.969}\color{fgcolor}\begin{kframe}
\begin{alltt}
\hlcomment{##################}
\hlcomment{# Source code file used to create Figure 3 in ''My Article''}
\hlcomment{# Created by Christopher Gandrud}
\hlcomment{# Updated 1 March 2012}
\hlcomment{##################}
\end{alltt}
\end{kframe}
\end{knitrout}


\noindent Feel free to use things like the long series of hash marks above and below the header, white space, and indentations to make your comments more readable. 

\paragraph{Style guides}
In natural language writing you don't necessarily need to always follow a style guide\index{style guide}. People could probably figure out what you are saying, but it would be a lot easier for your readers if you use consistent rules. The same is true when writing computer code. It's good to follow consistent rules for formatting your code so that its easier for you and others to understand.

There are a number of R style guides. Most of them are similar to the Google R Style Guide\index{Google R Style Guide}.\footnote{See: \url{http://google-styleguide.googlecode.com/svn/trunk/google-r-style.html}.} Hadley Wickham also has a nicely presented R style guide.\footnote{You can find it at \url{https://github.com/hadley/devtools/wiki/Style}.} You may want to use the {\emph{formatR}}\index{formatR} \cite[]{R-formatR} package to automatically reformat your code so that it is easier to read.

\paragraph{Literate programming}

For particularly important pieces of research code it may be useful to not only comment on the source file, but also display code in presentation text. For example, you may want to include key parts of the code you used for your main statistical models and an explanation of this code in an appendix following your article. This is commonly referred to as literate programming \index{literate programming} \cite[]{Knuth1992}. 

\subsection{Explicitly tie your files together}

If everything is just a text file then research projects can be thought of as individual text files that have a relationship with one another. They are tied together. A data file is used as input for an analysis file. The results of an analysis are shown and discussed in a markup file that is used to create a PDF document. Researchers often do not explicitly document the relationships between files that they used in their research. For example, the results of an analysis--a table or figure--may be copied and pasted into a presentation document. It will be very difficult for future researchers to trace the table or figure back to a particular statistical model and a particular data set. Therefore, it is important to make the links between your files explicit. 

Tie commands are the most dynamic way to explicitly link your files together.\index{tie commands} These commands instruct the computer program you are using to use information from another file. In Table \ref{TableTieCommands} I have compiled a selection of key tie commands you will learn how to use in this book. We'll discuss many more, but these are some of the most important.

\begin{table}
    \caption{A Selection of Commands for Tying Together Your Research Files}
    \label{TableTieCommands}
    \vspace{0.3cm}
    {\footnotesize{
    \begin{tabular}{p{2.5cm} c p{5.25cm} p{2cm}}
        \hline \vspace{0.15cm}
        Command/Package/ Program & Language & Description & Chapters for Further Information \\[0.3cm]  
        \hline \hline
        {\emph{knitr}} & R & R package with commands for tying analysis code into presentation documents including those written in LaTeX and Markdown. & Used throughout See Table \ref{ChunkOptionsTable}. \\[0.25cm]
        {\tt{read.table}} & R & Reads a table into R. You can use this to import plain-text file formated data into R. & \hfill\ref{DataGather} \\[0.25cm]
        {\tt{read.csv}} & R & Same as \texttt{read.table} with default arguments set to import \texttt{.csv} formatted data files. & \hfill\ref{DataGather} \\[0.25cm]  
        API-based packages & R & Various packages use APIs to gather data from the internet. & \hfill\ref{DataGather} \\[0.25cm]
        {\tt{merge}} & R & Merges together data frames. & \hfill\ref{DataClean} \\[0.25cm]
        {\tt{source}} & R & Runs an R source code file. & \hfill\ref{StatsModel} \\[0.25cm]
        {\tt{source\_url}} & R & From the {\emph{devtools}} package. Runs an R source code file from a secure ({\tt{https}}) url like those used by GitHub & \hfill\ref{StatsModel} \\[0.25cm]
        {\tt{print(xtable())}} & R & Combining the \texttt{print} \& \texttt{xtable} commands creates LaTeX \& HTML tables from R objects & \hfill\ref{TablesChapter} \\[0.25cm]
        {\tt{toLaTeX}} & R & Converts R objects to LaTeX & \hfill\ref{GettingStartedRR} \\[0.25cm]
        {\tt{input}} & LaTeX & Includes LaTeX files inside of other LaTeX files & \hfill\ref{LargeDocs} \\[0.25cm]
        {\tt{include}} & LaTeX & Similar to {\tt{input}}, but puts page breaks on either side of the \texttt{included}-ed text. Usually it is used for including chapters. & \hfill\ref{LargeDocs} \\[0.25cm]
        {\tt{includegraphics}} & LaTeX & Inserts a figure into a LaTeX document. & \hfill\ref{FiguresChapter} \\[0.25cm]
        \texttt{![]()} & Markdown & Inserts a figure into a Markdown document. & \hfill\ref{MarkdownChapter} \\  [0.25cm]   
        Pandoc & Shell & A shell program for converting files from one markup language to another. Allows you to tie presentation documents together. & \hfill\ref{LargeDocs} \& \ref{MarkdownChapter} \\[0.25cm]   
        Make & Shell & A shell program for automatically building many files. & \ref{DataGather} \\[0.25cm]
        \hline 
    \end{tabular}
    }}
\end{table}

\subsection{Have a plan to organize, store, \& make your files available}

Finally, in order for independent researchers to reproduce your work they need to be able access the files that instruct them how to do this. Files also need to be organized so that independent researchers can figure out how they fit together. So, from the beginning of your research process you should have a plan for organizing your files and a way to make them accessible. 

One rule of thumb for organizing your research in files is to limit the amount of content any one file has. Files that contain many different operations can be very difficult to navigate, even if they have detailed comments. For example, it would be very difficult to find any particular operation in a file that contained the code used to gather the data, run all of the statistical models, and create the results figures and tables. If you have a hard time finding things in a file you created, think of the difficulties independent researchers will have! 

Because we have so many ways to link files together there is really no need to lump many different operations into one file. So, we can make our operations modular. One source code file should be used to complete one task. Breaking your operations into discrete parts will also make it easier for you and others to find errors \cite[490]{Nagler1995}.

Chapter \ref{DirectoriesChapter} discusses file organization in much more detail. Chapter \ref{Storing} teaches you a number of ways to make your files accessible through cloud computing services like Dropbox\index{Dropbox} and GitHub\index{GitHub}.


% Chapter Chapter 3 For Reproducible Research in R and RStudio
% Christopher Gandrud
% Created: 16/07/2012 05:45:03 pm CEST
% Updated: 16 January 2013




\chapter{Getting Started with R, RStudio, and knitr}\label{GettingStartedRKnitr}

If you have rarely or never used R before, the first section of this chapter gives you enough information to be able to get started and understand the R code I use in this book. For more detailed introductions on how to use R please refer to the resources I mentioned in Chapter \ref{Intro}. Experienced R users might want to skip the first section. In the second section I'll give a brief overview of RStudio. I highlight the key features of the main RStudio panel (what appears when you open RStudio) and some of its key features for reproducible research. Finally, I discuss the basics of the {\emph{knitr}} package, how to use it in R, and how it is integrated into RStudio.

%%%%%%%%%%%%% Using R
\section{Using R: the basics}

To get you started with reproducible research, we'll cover some very basic R syntax--the rules for talking to R. I cover key parts of R including:

\begin{itemize}
    \item objects \& assignment,
    \item component selection,
    \item functions and commands,
    \item arguments,
    \item the workspace and history,
    \item libraries.
\end{itemize}

Before discussing each of these in detail let's open R and look around.\footnote{Please see Chapter \ref{Intro} for instructions on how to install R.} When you open the R GUI program you should get a window that looks something like Figure \ref{RBlankMain}.\footnote{This figure and almost all screenshots in this book were taken on a computer using the Mac OS 10.8 operating system.} This window is the {\bf{R console}}\index{R!Console}. After the startup information--information about what version of R you are using, license details, and so on--you should see a {\tt{\textgreater}}. This prompt is where you enter R code.\footnote{If you are using a Unix-like\index{Unix} system such as Ubuntu\index{Ubuntu} or Mac OS 10\index{Mac}, you can also access R via an application called the Terminal\index{Terminal}. If you have installed R on your computer you can type {\tt{r}} into the Terminal and then the {\tt{Enter}} or {\tt{Return}} key. This will begin a new R session. You know if a new R session has started if you get the same startup information is printed in the Terminal window.} To run R code that you have typed after the prompt hit the {\tt{Enter}} or {\tt{Return}} key. Now that we have a new R session\index{R!session} open we can get started. 

\begin{figure}[th!]
    \caption{R Startup Console}
    \label{RBlankMain}
    \begin{center}
    \includegraphics[scale=0.4]{Children/Chapter3/images3/BlankRConsole.png}
    \end{center}
\end{figure}

\subsection{Objects}\label{Objects}

If you've read a description of R before, you will probably have seen it referred to as an `object-oriented\index{object-oriented} language'.  What are objects? Objects are like the R language's nouns. They are things, like a vector of numbers, a data set, a word, a table of results from some analysis, and so on. Saying that R is `object-oriented' just means that R is focused on doing actions to objects. We will talk about the actions--commands and functions--later in this section.\footnote{Somewhat confusingly, commands and functions are themselves objects. In this Chapter I treat them as distinct from other object types.} For now let's create a few objects.

\paragraph{Numeric \& string objects}

Objects can have a number of different data types. Let's make two simple objects. The first is a numeric type object. The other is a character object. We can choose almost any name\label{ObjectNames} we want for our objects as long as it begins with an alphabetic character and does not contain spaces.\footnote{It is common for people to use either periods (\texttt{.}) or capital letters (referred to as CamelBack) to separate words in object names instead of using spaces. For example: {\emph{new.data}} or {\emph{NewData}} rather than {\emph{new data}}.} Let's call our numeric object {\emph{Number}}. It is a good idea to give each object a different name. Also make sure that object names are different from variable names. This will avoid many complications like accidentally overwriting an object or confusing R about what object or component you are referring to.

To put something into the object we use the assignment operator\index{R!assigmnent operator}\footnote{The assignment operator is sometimes also referred to as the `gets arrow'.}: {\tt{\textless -}}. Let's assign the number 10 to our {\emph{Number}} object.

\begin{knitrout}
\definecolor{shadecolor}{rgb}{0.969, 0.969, 0.969}\color{fgcolor}\begin{kframe}
\begin{alltt}
Number <- 10
\end{alltt}
\end{kframe}
\end{knitrout}


\noindent To see the contents of our object, type its name.

\begin{knitrout}
\definecolor{shadecolor}{rgb}{0.969, 0.969, 0.969}\color{fgcolor}\begin{kframe}
\begin{alltt}
Number
\end{alltt}
\begin{verbatim}
## [1] 10
\end{verbatim}
\end{kframe}
\end{knitrout}


\noindent Lets's briefly breakdown this output. \texttt{10} is clearly the contents of {\emph{Number}}. The double hash (\texttt{\#\#}) is included to tell you that this is output rather than R code.\footnote{The double hash is generated automatically by {\emph{knitr}}. It makes easier to copy and past code into R from a presentation document by {\emph{knitr}}.} If you type the commands in your R Console, you will not get the double hash in your output. Finally, \texttt{[1]} is the row number of the object that 10 is on. Clearly our object only has one row.   

Creating a character object is very similar. The only difference is that you enclose the character string (letters in a word for example) inside of single or double quotation marks (\verb|''|, or \verb|""|).\footnote{Single and double quotation marks are interchangable in R for this purpose. In this book I always use double quotes, except for \emph{knitr} code chunk options.} To create an object called \emph{Words} that contains the character string ``Hello World".\label{StringObject}

\begin{knitrout}
\definecolor{shadecolor}{rgb}{0.969, 0.969, 0.969}\color{fgcolor}\begin{kframe}
\begin{alltt}
Words <- \hlstring{"Hello World"}
\end{alltt}
\end{kframe}
\end{knitrout}


An object's type is important to keep in mind as it determines what we can do to it. For example, you cannot take the mean of a character object like the {\emph{Words}} object we created earlier:

\begin{knitrout}
\definecolor{shadecolor}{rgb}{0.969, 0.969, 0.969}\color{fgcolor}\begin{kframe}
\begin{alltt}
\hlfunctioncall{mean}(Words)
\end{alltt}


{\ttfamily\noindent\color{warningcolor}{\#\# Warning: argument is not numeric or logical: returning NA}}\begin{verbatim}
## [1] NA
\end{verbatim}
\end{kframe}
\end{knitrout}


\noindent Trying to find the mean of our {\emph{Words}} object gave us a warning message and returned the value {\tt{NA}}\index{NA}: not applicable. You can also think of {\tt{NA}} as meaning missing. To find out what type of object you have use the {\tt{class}} command. For example:

\begin{knitrout}
\definecolor{shadecolor}{rgb}{0.969, 0.969, 0.969}\color{fgcolor}\begin{kframe}
\begin{alltt}
\hlfunctioncall{class}(Words)
\end{alltt}
\begin{verbatim}
## [1] "character"
\end{verbatim}
\end{kframe}
\end{knitrout}


\paragraph{Vector \& data frame objects}

So far we have only looked at objects with a single number or character string.\footnote{These might be called scalar objects, though in R scalars are just vectors with a length of 1.} Clearly we often want to use objects that have many strings and numbers. In R these are usually data frame\index{data frame} type objects and are roughly equivalent the data structures you would be familiar with from using a program such as Microsoft Excel. We will be using data frames extensively throughout the book. Before looking at data frames it is useful to first look at the simpler objects that make up data frames. These are called vectors. Vectors are R's ``workhorse" \cite[]{Matloff2011}. Knowing how to use vectors\index{vector} will be especially helpful when you clean up raw data in Chapter \ref{DataClean} and make tables in Chapter \ref{TablesChapter}.\footnote{If you want information about other types of R objects such as lists\index{list} and matrices\index{matrix}, Chapter 1 of Norman Matloff's \citeyearpar{Matloff2011} book is a really good place to look.} \\[0.25cm]

\noindent {\bf{Vectors}} \\[0.25cm] Vectors are the ``fundamental data type" in R \cite[]{Matloff2011}. They are simply an ordered group of numbers, character strings, and so on.\footnote{In a vector every member of the group must be of the same type. If you want an ordered group of values with different types you can use lists.} It may be useful to think of basically all R objects as composed of vectors. For example, data frames are basically multiple vectors of the same length--i.e. they have the same number of rows--attached together to form columns. 

Let's create a simple numeric vector containing the numbers 2.8, 2, and 14.8. To do this we will use the \texttt{c} (concatenate)\index{concatenate}\index{R command!c} function:

\begin{knitrout}
\definecolor{shadecolor}{rgb}{0.969, 0.969, 0.969}\color{fgcolor}\begin{kframe}
\begin{alltt}
NumericVect <- \hlfunctioncall{c}(2.8, 2, 14.8)

\hlcomment{# Show NumericVect's contents}
NumericVect
\end{alltt}
\begin{verbatim}
## [1]  2.8  2.0 14.8
\end{verbatim}
\end{kframe}
\end{knitrout}


Vectors of character strings are created in a similar way. The only major difference is that each character string is enclosed in quotation marks like this:

\begin{knitrout}
\definecolor{shadecolor}{rgb}{0.969, 0.969, 0.969}\color{fgcolor}\begin{kframe}
\begin{alltt}
CharacterVect <- \hlfunctioncall{c}(\hlstring{"Albania"}, \hlstring{"Botswana"}, \hlstring{"Cambodia"})

\hlcomment{# Show CharacterVect's contents}
CharacterVect
\end{alltt}
\begin{verbatim}
## [1] "Albania"  "Botswana" "Cambodia"
\end{verbatim}
\end{kframe}
\end{knitrout}


To give you a preview of what we are going to do when we start working with real data sets, lets combine the two vectors {\emph{NumericVect}} and {\emph{CharacterVect}} into a new object with the \texttt{cbind}\index{R command!cbind} function. This function binds the two vectors together side-by-side as columns.\footnote{If you want to combine objects as if they were rows of the same column(s) use the \texttt{rbind}\index{rbind} function.}\label{cbind}

\begin{knitrout}
\definecolor{shadecolor}{rgb}{0.969, 0.969, 0.969}\color{fgcolor}\begin{kframe}
\begin{alltt}
StringNumObject <- \hlfunctioncall{cbind}(CharacterVect, NumericVect)

\hlcomment{# Show StringNumObject's contents}
StringNumObject
\end{alltt}
\begin{verbatim}
##      CharacterVect NumericVect
## [1,] "Albania"     "2.8"      
## [2,] "Botswana"    "2"        
## [3,] "Cambodia"    "14.8"
\end{verbatim}
\end{kframe}
\end{knitrout}


\noindent By binding these two objects together we've created a new matrix\index{matrix} object.\footnote{Matrices are vectors with columns as well as rows.} You can see that the numbers in the {\emph{NumericVect}} column are between quotation marks. Matrices, like vectors can only have one data type. \\[0.25cm]

\noindent {\bf{Data frames}}

If we want to have an object with rows and columns and allow the columns to contain data with different types, we need to use data frames\index{data frame}. Let's use the \texttt{data.frame}\index{R command!data.frame} command to combine the {\emph{NumericVect}} and {\emph{CharacterVect}} objects.\label{data.frame}

\begin{knitrout}
\definecolor{shadecolor}{rgb}{0.969, 0.969, 0.969}\color{fgcolor}\begin{kframe}
\begin{alltt}
StringNumObject <- \hlfunctioncall{data.frame}(CharacterVect, NumericVect)

\hlcomment{# Display contents of StringNumObject data frame}
StringNumObject
\end{alltt}
\begin{verbatim}
##   CharacterVect NumericVect
## 1       Albania         2.8
## 2      Botswana         2.0
## 3      Cambodia        14.8
\end{verbatim}
\end{kframe}
\end{knitrout}


\noindent There are a few important things to notice in this output. The first is that because we used the same name for the data frame object as the previous matrix object, R deleted the matrix object and replaced it with the data frame. This is something to keep in mind when you are creating new objects. In general it is a better idea to assign elements to new objects rather than overwriting old ones. This will help avoid accidentally using an object you had not intended to. It also allows you to more easily change previously run source code.

You can see the data frame's names\index{R command!names} attribute.\footnote{Matrices can also have a names attribute.} It is the column names. You can use the \texttt{names} command to see any data frame's names:\footnote{You can also use \texttt{names} to assign names for the entire data frame. For example, \texttt{names(StringNumObject) <- c("Variable1", "Variable2")}}

\begin{knitrout}
\definecolor{shadecolor}{rgb}{0.969, 0.969, 0.969}\color{fgcolor}\begin{kframe}
\begin{alltt}
\hlfunctioncall{names}(StringNumObject)
\end{alltt}
\begin{verbatim}
## [1] "CharacterVect" "NumericVect"
\end{verbatim}
\end{kframe}
\end{knitrout}


You will also notice that the first column of the data set has no name and is a series of numbers. This is the row.names attribute.\index{row.names} Unlike with matrices, data frame rows can be given any name as long as they are unique. We can use the \texttt{row.names} command to set the row names from a vector. For example,

\begin{knitrout}
\definecolor{shadecolor}{rgb}{0.969, 0.969, 0.969}\color{fgcolor}\begin{kframe}
\begin{alltt}
\hlcomment{# Reassign row.names}
\hlfunctioncall{row.names}(StringNumObject) <- \hlfunctioncall{c}(\hlstring{"First"}, \hlstring{"Second"}, \hlstring{"Third"})

\hlcomment{# Display new row.names}
\hlfunctioncall{row.names}(StringNumObject)
\end{alltt}
\begin{verbatim}
## [1] "First"  "Second" "Third"
\end{verbatim}
\end{kframe}
\end{knitrout}


You can see in this example how the \texttt{row.names} command can also be used to see the row names.\footnote{Note that this is really only useful for data frames with few rows.} The row.names attribute does not behave like a regular data frame column. You cannot, for example include it as a variable in a regression. You can use the \emph{row.names} command to assign the row.names values to a regular column (for an example see page \pageref{RowNamesTidy}). 

You will notice in the output for \emph{StringNumObject} that the strings in the \textbf{CharacterVect} column are no longer in quotation marks. This does not mean that they are somehow now numeric data. To prove this try to find the mean of \textbf{CharacterVect} by running it through the \texttt{mean}\index{R command!mean} command:

\begin{knitrout}
\definecolor{shadecolor}{rgb}{0.969, 0.969, 0.969}\color{fgcolor}\begin{kframe}
\begin{alltt}
\hlfunctioncall{mean}(StringNumObject$ChacterVect)
\end{alltt}


{\ttfamily\noindent\color{warningcolor}{\#\# Warning: argument is not numeric or logical: returning NA}}\begin{verbatim}
## [1] NA
\end{verbatim}
\end{kframe}
\end{knitrout}


\subsection{Component Selection}

The last bit of code will probably be confusing. Why do we have a dollar sign (\texttt{\$}) in between the name of our data frame object and the \textbf{CharcterVect}? The dollar sign is called the component selector.\index{component selector}\index{R!\$, component selector}\footnote{It's also sometimes called the element name operator.} It basically extracts a part--component--of an object. In the previous example it extracted the \textbf{CharacterVect} column from the {\emph{StringNumObject}} and fed it to the \texttt{mean} command, which tried (in this case unsuccessfully) to find its mean.

We can of course use the component selector to create new objects with parts of other objects. Imagine that we have the {\emph{StringNumObject}} and want an object with only the information in the numbers column. Let's use the following code:

\begin{knitrout}
\definecolor{shadecolor}{rgb}{0.969, 0.969, 0.969}\color{fgcolor}\begin{kframe}
\begin{alltt}
NewNumeric <- StringNumObject$NumericVect

\hlcomment{# Display contents of NewNumeric}
NewNumeric
\end{alltt}
\begin{verbatim}
## [1]  2.8  2.0 14.8
\end{verbatim}
\end{kframe}
\end{knitrout}


\noindent Knowing how to use the component selector will be especially useful when we discuss making tables for presentation documents in Chapter \ref{TablesChapter}.

\paragraph{{\tt{attach}} and {\tt{with}}}

Using the component selector can create long repetitive code. You have to write the object name, a dollar sign, and the component name every time you want to select a component. You can streamline your code by using commands such as \texttt{attach}\index{R command!attach} and \texttt{with}\index{R command!with}. 

The \texttt{attach} command attaches a database to R's search path.\footnote{You can see what is in your current search path with the \texttt{search}\index{R command!search} command. Just type \texttt{search()} into your R console.}\index{R!search path} R will then search the database for variables you specify. You don't need to use the component selector to tell R again to look in a particular data frame after you have attached it. For example, let's attach the \emph{cars} data, that comes with R. It has two variables \textbf{speed} and \textbf{dist}.\footnote{For more information on this data set type \texttt{?cars} into your R console.}

{\small
\begin{knitrout}
\definecolor{shadecolor}{rgb}{0.969, 0.969, 0.969}\color{fgcolor}\begin{kframe}
\begin{alltt}
\hlcomment{# Attach cars to search path}
\hlfunctioncall{attach}(cars)

\hlcomment{# Display speed}
\hlfunctioncall{head}(speed)
\end{alltt}
\begin{verbatim}
## [1] 4 4 7 7 8 9
\end{verbatim}
\begin{alltt}

\hlcomment{# Display dist}
\hlfunctioncall{head}(dist)
\end{alltt}
\begin{verbatim}
## [1]  2 10  4 22 16 10
\end{verbatim}
\begin{alltt}

\hlcomment{# Detach cars}
\hlfunctioncall{detach}(cars)
\end{alltt}
\end{kframe}
\end{knitrout}

}

\noindent We used the \texttt{head}\index{R command!head} command to see just the top of each variable. It is a good idea to \texttt{detach}\index{R command!detach} a data frame after you are done using it, to avoid confusing R. 

Similarly you can use \texttt{with} to run a command (see page \pageref{FunctionsCommands} for more details) using a particular database. For example, we can find the mean of \emph{NumericVect} \texttt{with} the \emph{StringNumObject} data frame:

\begin{knitrout}
\definecolor{shadecolor}{rgb}{0.969, 0.969, 0.969}\color{fgcolor}\begin{kframe}
\begin{alltt}
\hlfunctioncall{with}(StringNumObject, \{
    \hlfunctioncall{mean}(NumericVect)
    \}
)
\end{alltt}
\begin{verbatim}
## [1] 6.533
\end{verbatim}
\end{kframe}
\end{knitrout}


\noindent You can see in the \texttt{with} command the data frame object goes first and then the \texttt{mean}\index{R command!mean} command\footnote{More appropriately the second, part or ``argument"\index{R!argument} (see page \pageref{arguments}) is called the ``expression"\index{expression, R} because it can contain more than one R command or statement.} goes second in curly brackets (\verb|{}|).

For examples in this book I largely avoid using the \texttt{attach} and \texttt{with} commands. I mostly use the component selector. Though it creates longer code, I find that code written with the component selector is easier to follow. It's always clear which object we are selecting a component from. Nonetheless, they can be very useful for streamlining your R code.

\subsection{Subscripts}

Another way to select parts of an object is to use subscripts\index{subscripts}. You have already seen subscripts in the output from our examples so far. They are denoted with square braces (\texttt{[]}). We can use subscripts to select not only columns from data frames but also rows and individual cells. As we began to see in some of the previous output, each part of a data frame has an address captured by its row and column number. We can tell R to find a part of an object by putting the row number/name, column number/name, or both in square braces. The first part denotes the rows and separated by a comma (\texttt{,}) are the columns. 

To give you an idea of how this works lets use the {\emph{cars}} data set that comes with R. Use the \texttt{head} command to get a sense of what this data set looks like.

\begin{knitrout}
\definecolor{shadecolor}{rgb}{0.969, 0.969, 0.969}\color{fgcolor}\begin{kframe}
\begin{alltt}
\hlfunctioncall{head}(cars)
\end{alltt}
\begin{verbatim}
##   speed dist
## 1     4    2
## 2     4   10
## 3     7    4
## 4     7   22
## 5     8   16
## 6     9   10
\end{verbatim}
\end{kframe}
\end{knitrout}


\noindent We can see a data frame with information on various cars speeds (\textbf{speed}) and stopping distances (\textbf{dist}). If we want to select only the third through seventh rows we can use the following subscript commands:

\begin{knitrout}
\definecolor{shadecolor}{rgb}{0.969, 0.969, 0.969}\color{fgcolor}\begin{kframe}
\begin{alltt}
cars[3:7, ]
\end{alltt}
\begin{verbatim}
##   speed dist
## 3     7    4
## 4     7   22
## 5     8   16
## 6     9   10
## 7    10   18
\end{verbatim}
\end{kframe}
\end{knitrout}


\noindent The colon (\texttt{:}) creates a sequence of whole numbers from 3 to 7. To select the fourth row of the \textbf{dist} column we can type:

\begin{knitrout}
\definecolor{shadecolor}{rgb}{0.969, 0.969, 0.969}\color{fgcolor}\begin{kframe}
\begin{alltt}
cars[4, 2]
\end{alltt}
\begin{verbatim}
## [1] 22
\end{verbatim}
\end{kframe}
\end{knitrout}


\noindent An equivalent way to do this is:

\begin{knitrout}
\definecolor{shadecolor}{rgb}{0.969, 0.969, 0.969}\color{fgcolor}\begin{kframe}
\begin{alltt}
cars[4, \hlstring{"dist"}]
\end{alltt}
\begin{verbatim}
## [1] 22
\end{verbatim}
\end{kframe}
\end{knitrout}


\noindent Finally, we can even include a vector of column names to select:

\begin{knitrout}
\definecolor{shadecolor}{rgb}{0.969, 0.969, 0.969}\color{fgcolor}\begin{kframe}
\begin{alltt}
cars[4, \hlfunctioncall{c}(\hlstring{"speed"}, \hlstring{"dist"})]
\end{alltt}
\begin{verbatim}
##   speed dist
## 4     7   22
\end{verbatim}
\end{kframe}
\end{knitrout}


\subsection{Functions and commands}\label{FunctionsCommands}

If objects are the nouns of the R language, functions and commands\footnote{For the purposes of this book I treat the two as the same.} are the verbs. They do things to objects. Let's use the \texttt{mean} command as an example. This command takes the mean of a numeric vector object. Remember our {\emph{NumericVect}} object from before:

\begin{knitrout}
\definecolor{shadecolor}{rgb}{0.969, 0.969, 0.969}\color{fgcolor}\begin{kframe}
\begin{alltt}
\hlcomment{# Show contents of NumericVect}
NumericVect
\end{alltt}
\begin{verbatim}
## [1]  2.8  2.0 14.8
\end{verbatim}
\end{kframe}
\end{knitrout}


\noindent To find the mean of this object simply type:

\begin{knitrout}
\definecolor{shadecolor}{rgb}{0.969, 0.969, 0.969}\color{fgcolor}\begin{kframe}
\begin{alltt}
\hlfunctioncall{mean}(x = NumericVect)
\end{alltt}
\begin{verbatim}
## [1] 6.533
\end{verbatim}
\end{kframe}
\end{knitrout}


\noindent We use the assignment operator to place a command's output into an object. For example:

\begin{knitrout}
\definecolor{shadecolor}{rgb}{0.969, 0.969, 0.969}\color{fgcolor}\begin{kframe}
\begin{alltt}
MeanNumericVect <- \hlfunctioncall{mean}(x = NumericVect)
\end{alltt}
\end{kframe}
\end{knitrout}


\noindent Notice that we typed the command's name then enclosed the object name in parentheses immediately afterwards. This is the basic sytax that all commands use, i.e. \texttt{COMMAND(ARGUMENTS)}. If you don't want to explicitly include an argument you still need to type the parentheses after the command. 

\subsection{Arguments}\label{arguments}

Arguments\index{R!command argument} modify what commands do. In our most recent example we gave the \texttt{mean} command one argument (\texttt{x = NumericVect}) telling it that we wanted to find the mean of {\emph{NumericVect}}. Arguments use the \texttt{ARGUMENTLABEL = VALUE} syntax.\footnote{Note: you do not have to put spaces between the argument label and the equals sign or the equals sign and the value. However, having spaces can make your code easier for other people to read.} 

To find all of the arguments that an argument can accept look at the {\bf{Arguments}} section of the command's help file\index{help file}. To access the help file type: \texttt{?COMMAND}. For example,

\begin{knitrout}
\definecolor{shadecolor}{rgb}{0.969, 0.969, 0.969}\color{fgcolor}\begin{kframe}
\begin{alltt}
?mean
\end{alltt}
\end{kframe}
\end{knitrout}


\noindent The help file will also tell you the default values that the arguments are set to. Clearly, you do not need to explicitly set an argument if you want to use it's default value.

You have to fairly precise with the syntax for your argument's values. Arguments for logical arguments must written as \texttt{TRUE} or \texttt{FALSE}.\footnote{They can be abbreviated \texttt{T} and \texttt{F}.} Arguments that are character strings should be in quotation marks.

Let's see how to use multiple arguments with the \texttt{round} command. This command rounds a vector of numbers. We can use the \texttt{digits} option to specify how many decimal places we want the numbers rounded to. To round the object \emph{MeanNumericVect} to one decimal place type:

\begin{knitrout}
\definecolor{shadecolor}{rgb}{0.969, 0.969, 0.969}\color{fgcolor}\begin{kframe}
\begin{alltt}
\hlfunctioncall{round}(x = MeanNumericVect, digits = 1)
\end{alltt}
\begin{verbatim}
## [1] 6.5
\end{verbatim}
\end{kframe}
\end{knitrout}


\noindent You can see that arguments are separated by commas. 

Some arguments do not need to be explicitly labeled. For example we could have written:

\begin{knitrout}
\definecolor{shadecolor}{rgb}{0.969, 0.969, 0.969}\color{fgcolor}\begin{kframe}
\begin{alltt}
\hlcomment{# Find mean of NumericVect}
\hlfunctioncall{mean}(NumericVect)
\end{alltt}
\begin{verbatim}
## [1] 6.533
\end{verbatim}
\end{kframe}
\end{knitrout}


\noindent R will do its best to figure out what you want and will only give up when it can't. This will generate an error message. However, to avoid any misunderstandings between yourself and R it can be good practice to label all of your arguments. This will also make your code easier for other people to read, i.e. it will be more reproducible.

Finally, you can stack arguments inside of other arguments. To have R find the mean of {\emph{NumericVect}} and round it to one decimal place use:

\begin{knitrout}
\definecolor{shadecolor}{rgb}{0.969, 0.969, 0.969}\color{fgcolor}\begin{kframe}
\begin{alltt}
\hlfunctioncall{round}(\hlfunctioncall{mean}(NumericVect), digits = 1)
\end{alltt}
\begin{verbatim}
## [1] 6.5
\end{verbatim}
\end{kframe}
\end{knitrout}


\subsection{The Workspace \& History}

All of the objects you create become part of your workspace\index{workspace}. Use the \texttt{ls}\index{shell command!ls} command to list all of the objects in your current workspace.\footnote{Note: your workspace will probably include different objects than this example. These are all of the objects created to knit the book up to this point.}

\begin{knitrout}
\definecolor{shadecolor}{rgb}{0.969, 0.969, 0.969}\color{fgcolor}\begin{kframe}
\begin{alltt}
\hlfunctioncall{ls}()
\end{alltt}
\begin{verbatim}
## [1] "CharacterVect"   "doInstall"       "MeanNumericVect" "NewNumeric"     
## [5] "Number"          "NumericVect"     "StringNumObject" "toInstall"      
## [9] "Words"
\end{verbatim}
\end{kframe}
\end{knitrout}


You can remove specific objects from the workspace using the \texttt{rm} command\index{rm}. For example, to remove the \texttt{CharacterVect} and \texttt{Words} objects type:

\begin{knitrout}
\definecolor{shadecolor}{rgb}{0.969, 0.969, 0.969}\color{fgcolor}\begin{kframe}
\begin{alltt}
\hlfunctioncall{rm}(CharacterVect, Words)
\end{alltt}
\end{kframe}
\end{knitrout}


To save the entire workspace into a binary--not plain-text--RData file use the \texttt{save.image}\index{save.image}\label{SaveLoadWS} command. The main argument of the \texttt{save.image} command is the file path you would like the file saved into. If you don't specify the file path it will save in your current working directory (see Chapter \ref{DirectoriesChapter}). For example, to save the current workspace in a file called \emph{DecemberWorkspace.RData} in the current working directory type:

\begin{knitrout}
\definecolor{shadecolor}{rgb}{0.969, 0.969, 0.969}\color{fgcolor}\begin{kframe}
\begin{alltt}
\hlfunctioncall{save.image}(file = \hlstring{"DecemberWorkspace.RData"})
\end{alltt}
\end{kframe}
\end{knitrout}


\noindent Use the \texttt{load}\index{R command!load} to load a saved workspace back into R:

\begin{knitrout}
\definecolor{shadecolor}{rgb}{0.969, 0.969, 0.969}\color{fgcolor}\begin{kframe}
\begin{alltt}
\hlfunctioncall{load}(file = \hlstring{"DecemberWorkspace.RData"})
\end{alltt}
\end{kframe}
\end{knitrout}


You should generally avoid having R save your workspace when you quit and reload it when you start R again. Instead, when you return to working on a project rerun the source code files. This avoids any complications caused when you use an object in your workspace that is left over from running an older version of the source code.\footnote{For example, imagine you create an object, then change the source code you used to create the object. However, there is a syntax error in the new version of the source code. The old object won't be overwritten and you will be mistakenly using the old object in future commands.} 

The only time when saving your workspace is very useful is when it includes an object that was computationally difficult and took a long time to create. In this case you can save only the large object with the \texttt{save} command.\footnote{The \texttt{save.image} command is just a special case of \texttt{save}.}\index{R command!save}\label{RSave} For example, if we have a very large object called \emph{Comp} we can save it to a file called \emph{Comp.RData} like this:

\begin{knitrout}
\definecolor{shadecolor}{rgb}{0.969, 0.969, 0.969}\color{fgcolor}\begin{kframe}
\begin{alltt}
\hlfunctioncall{save}(Comp, file = \hlstring{"Comp.RData"})
\end{alltt}
\end{kframe}
\end{knitrout}


In general I also recommend against saving data in binary RData formatted files. Because they are not text files they are not human readable and are much less future proof.

\paragraph{R history}

When you enter a command into R it becomes part of your history. To see the most recent commands in your history use the \texttt(history) command. You can also use the up and down arrows on your keyboard when your cursor is in the R console to scroll through your history.

\subsection{Global R options}\label{ROptions}

In R you can set global options with the \texttt{options} command\index{R command!options} This lets you set how R runs and outputs commands through an entire R session. For example, to have output rounded to one decimal place set set the \texttt{digits}\index{digits} argument:

\begin{knitrout}
\definecolor{shadecolor}{rgb}{0.969, 0.969, 0.969}\color{fgcolor}\begin{kframe}
\begin{alltt}
\hlcomment{# Round output to one decimal place}
\hlfunctioncall{options}(digits = 1)
\end{alltt}
\end{kframe}
\end{knitrout}


\subsection{Installing new packages and loading commands}\label{Packages}

Commands are stored in R packages\index{R!packages}. The commands we have used so far were loaded automatically by default. One of the great things about R is the many user-created packages\footnote{For the latest list see: \url{http://cran.r-project.org/web/packages/available_packages_by_name.html}} that greatly expand the number of commands we can use. To install commands that do not come with base R you need to install the add-on packages\label{packages} that contain them. To do this use the {\tt{install.packages}}\index{install.packages} command. By default this command downloads and installs the packages from the Comprehensive R Archive Network (CRAN)\index{CRAN}. 

For the code you need to install all of the packages used in this book see page \pageref{ReqPackages}. When you install a package, you will likely be given a list of mirrors\index{mirrors, CRAN} from which you can download the package. Simply select the mirror closest to you.

Once you have installed a package you need to load it so that you can use its functions. Use the \texttt{library} command to load a package.\footnote{You will probably see R packages referred to as ``libraries'', though this is a misnomer. See this blog post by Carlisle Rainey for a discussion: \url{http://www.carlislerainey.com/2013/01/02/packages-v-libraries-in-r/?utm_source=rss&utm_medium=rss&utm_campaign=packages-v-libraries-in-r} (posted 2 January 2013).} Use the following code to load the {\emph{ggplot2}} package that we use in Chapter \ref{FiguresChapter} to create figures.

\begin{knitrout}
\definecolor{shadecolor}{rgb}{0.969, 0.969, 0.969}\color{fgcolor}\begin{kframe}
\begin{alltt}
\hlfunctioncall{library}(ggplot2)
\end{alltt}
\end{kframe}
\end{knitrout}


\noindent Please note that for the examples in this book I only specify what package a command is from if the it is not loaded by default when you start an R session. 

Finally, if you want to make sure R uses a command from a specific package you can use the double-colon operator (\verb|::|).\index{R!double-colon operator} For example, to make sure that we use the \texttt{qplot}\index{qplot} command from \emph{ggplot2} we would type:

\begin{knitrout}
\definecolor{shadecolor}{rgb}{0.969, 0.969, 0.969}\color{fgcolor}\begin{kframe}
\begin{alltt}
ggplot2::\hlfunctioncall{qplot}(. . .)
\end{alltt}
\end{kframe}
\end{knitrout}


\noindent We can use the double-colon to simplify our code by removing the need to include \verb|library(. . .)|. Though, I generally prefer loading all of the packages used in a source code file at the begining with \texttt{library}. This makes it clearer right up front what packages are used.

%%%%%%%%%%%%%%% Using RStudio
\section{Using RStudio}

As I mentioned in Chapter \ref{Intro}, RStudio is an integrated development environment for R. It provides a centralized and well organized place to do almost anything you want to do with R. As we will see later in this chapter, it is especially well integrated with literate programming tools for reproducible research. Right now let's take a quick tour of the basic RStudio window.

\paragraph{The default window}

When you first open RStudio\index{RStudio} you should see a default window that looks like Figure \ref{BlankMain}. In this figure you see three window panes\index{RStudio!Pane}. The large one on the left is the {\emph{Console}}. This pane functions exactly the same as the console in regular R. Other panes include the {\emph{Workspace/History}} panes, usually in the upper right-hand corner. The Workspace pane shows you all of the objects in your current workspace and some of their characteristics, like how many observations a data frame has. You can click on an object in this pane to see its contents. This is especially useful for quickly looking at a data set in much the same way that you can visually scan a Microsoft Excel spreadsheet. The History pane records all of the commands you have run. It allows you to rerun code and insert it into a source code file.

\begin{figure}[ht]
    \caption{RStudio Startup Panel}
    \label{BlankMain}
    \begin{center}
    \includegraphics[width = \textwidth]{Children/Chapter3/images3/BlankMainPanel.png}
    \end{center}
\end{figure}

In the lower right-hand corner you will see the {\emph{Files/Plots/Packages/Help}} pane. We will discuss the Files pane in more detail in Chapter \ref{DirectoriesChapter}. Basically, it allows you to see and organize your files. The Plots pane is where figures you create in R appear. This pane allows you to see all of the figures you have created in a session using the right and left arrow icons. It also lets you save the figures in a variety of formats. The Packages pane shows the packages you have installed, allows you to load individual packages by clicking on the dialog box next to them, access their manual files (click on the package name), update the packages, and even install new packages. Finally, the Help pane shows you help files. You can search for help files and search within help files using this pane.  

\paragraph{The source pane}

There is an important pane that does not show up when you open RStudio for the first time. This is the {\emph{Source}} pane. The Source pane is where you create, edit, and run your source code files. It also functions as an editor for your markup files. It is the center of reproducible research in RStudio. Let's first look at how to use the Source pane with regular R files. We will cover how to use the Source pane with literate programming file formats--e.g. R Markdown and R LaTeX--in more detail after first discussing the {\emph{knitr}} basics in the next section. 

R source code files have the file extension \texttt{.R}. You can create a new source code document, which will open a new Source pane, by going to the menu bar and clicking on \texttt{File} \textrightarrow \: \texttt{New}. In this drop down menu you have the option to create a variety of different source code documents. Select the \texttt{R Source} option. You should now see a new pane with a bar across the top that looks like the first image in Figure \ref{SourcePanes}. To run the R code you have in your source code file simply highlight it\footnote{If you are only running one line of code you don't need to highlight the code, you can simply put your cursor on that line.} and click the \texttt{Run} icon on the top bar. This sends the code to the console where it is executed. The icon to the right of \texttt{Run} simply runs the code above where you have highlighted. The \texttt{Source} icon next to this runs all of the code in the file using R's \texttt{source} command\index{R command!soure}. The icon next to \texttt{Source} is for compiling RStudio Notebooks\index{RStudio!Notebook}. We will look at RStudio Notebooks later in this chapter.

\begin{figure}[ht]
    \caption{RStudio Source Code Pane Top Bars}
    \label{SourcePanes}
    \begin{center}
    
        \begin{subfigure}
            \caption{R Source Code}
            \includegraphics[width = \textwidth]{Children/Chapter3/images3/RSourceBar.png}
        \end{subfigure}
        \vspace{0.5cm}
        
        \begin{subfigure}
            \caption{R Markdown Files}
            \includegraphics[width = \textwidth]{Children/Chapter3/images3/MarkdownSourceBar.png}
        \end{subfigure}
        \vspace{0.5cm}
        
        \begin{subfigure}
            \caption{LaTeX Markdown Files}
            \includegraphics[width = \textwidth]{Children/Chapter3/images3/LaTeXSourceBar.png}
        \end{subfigure}
        
    \end{center}
\end{figure}



%%%%%%%%%%%%% Using knitr
\section{Using knitr: the basics}

To get started with {\emph{knitr}}\index{knitr} in R or RStudio we need to learn some of the basic concepts and syntax. The concepts are the same regardless of the markup language we are knitting R code with, but much of the syntax varies by markup language.

\subsection{What \emph{knitr} does}

Let's take a quick, abstract look at what the \emph{knitr} package does. As I've mentioned, \emph{knitr} ties together your presentation of results with the creation of those results. The \emph{knitr} process takes three steps (see Figure \ref{KnitProcess}). First we create a knitable markup document. This contains both the analysis code and the presentation document's markup--the text and rules for how to format the text. \emph{knitr} then \emph{knits}: i.e. it runs the analysis code and converts the output into the markup language you are using according to the rules that we tell it to. It inserts the marked-up results into a document that only contains markup for the presentation document. You \emph{compile}\index{compile markup} this markup document as you would if you hadn't used \emph{knitr} into your final PDF document or webpage presenting your results. 

\begin{figure}
    \caption{The {\emph{knitr}} Process}
    \label{KnitProcess}
        \begin{center}
            %%%%%%%%%%%%%%
% The knitr process
% Christopher Gandrud
% Updated 13 January 2012
%%%%%%%%%%%%%%

% Define colors for figure
%% Color palette (GnBU) chosen using ColorBrewer 2.0
%% See: http://colorbrewer2.org/
%% Not used in the print version
\definecolor{Blue}{HTML}{7BCCC4}
\definecolor{LiteBlue}{HTML}{A8DDB5}
\definecolor{DarkBlue}{HTML}{08589E}

\definecolor{GrayLine}{HTML}{BDBDBD}

% Set node styles
%% Workflow stage nodes
\tikzstyle{Docs} = [draw=Blue, 
                     rectangle, 
                     inner sep=0.3cm, 
                     font=\small]

% Begin tikz picture
\begin{tikzpicture}

	\node(knit) at (2, 1.75) {{\emph{\textbf{Knit}}}};
	\node(compile) at (6, 1.75) {{\emph{\textbf{Compile}}}};

	% Document nodes
	\node (knittable) at (0, 0) [Docs, text width= 6em]{Knittable Document \\ (Markup + Code Chunks)};
	\node (Markup) at (4, 0) [Docs, text width= 6em]{Markup Only Document};
	\node (Presentation) at (8, 0) [Docs, text width = 6em]{Presentation Document};

	% LaTeX Example
	\node(LaTeX) at (0, -2.5) {\textbf{LaTeX Example}};
	\node (Rnw) at (0, -3.5) [Docs, text width= 6em]{\emph{Paper.Rnw}};
	\node (tex) at (4, -3.5) [Docs, text width= 6em]{\emph{Paper.tex}};
	\node (pdf) at (8, -3.5) [Docs, text width = 6em]{\emph{Paper.pdf}};

	% Markdown Example
	\node(Markdown) at (0, -5) {\textbf{Markdown Example}};
	\node (Rmd) at (0, -6) [Docs, text width= 6em]{\emph{Website.Rmd}};
	\node (md) at (4, -6) [Docs, text width= 6em]{\emph{Website.md}};
	\node (html) at (8, -6) [Docs, text width = 6em]{\emph{Website.html}};

	% Lines
	\draw [->, very thick] (knittable) -- (Markup);
	\draw [->, very thick] (Markup) -- (Presentation);

	\draw [->, very thick] (Rnw) -- (tex);
	\draw [->, very thick] (tex) -- (pdf);

	\draw [->, very thick] (Rmd) -- (md);
	\draw [->, very thick] (md) -- (html);

\end{tikzpicture}
        \end{center}
\end{figure}

\subsection{File extensions}

When you save a knitable file use a file extension that indicates (a) that it is knitable and (b) what markup language it is using. You can use a number of file extensions for R Markdown files including: \texttt{.Rmd} and \texttt{.Rmarkdown}. LaTeX documents that include {\emph{knitr}} code chunks are generally called R Sweave\index{R Sweave} files and have the file extension {\tt{.Rnw}}. This terminology is a little confusing. It is a holdover from {\emph{knitr}}'s main literate programming predecessor {\emph{Sweave}} \cite[]{Leisch2002}. You can also use the less confusing file extension \texttt{.Rtex}, as regular LaTeX files have the extension \texttt{.tex}. However, the syntax for \texttt{.Rtex} files is different from that used with \texttt{.Rnw} files. We'll look at this issue in more detail below.

\subsection{Code Chunks}

When you want to include R code into your markup presentation documents, place them in a code chunk\index{code chunk}. Code chunk syntax differs depending on the markup language we are using to write our documents. Let's see the syntax for R Markdown and R LaTeX files.

\paragraph{R Markdown}

In R Markdown files we begin a code chunk by writing the head: \verb|```{r}|. A code chunk is closed--ended--simply with: \verb|```|. For example:

\begin{knitrout}
\definecolor{shadecolor}{rgb}{0.969, 0.969, 0.969}\color{fgcolor}\begin{kframe}
\begin{alltt}
```\{r\}
\hlcomment{# Example of an R Markdown code chunk}
StringNumObject <- \hlfunctioncall{cbind}(CharacterVect, NumericVect)
```
\end{alltt}
\end{kframe}
\end{knitrout}


\paragraph{R LaTeX}

There are two different ways to delimit code chunks in R LaTeX documents. One way largely emulates the established {\emph{Sweave}} syntax.\footnote{The syntax has its genesis in a literate programming tool called noweb \cite[]{Leisch2002,RamseyNoweb}.} {\emph{Knitr}} also supports files with the {\tt{.Rtex}} extension, though the code chunk syntax is different. I will cover both types of syntax for code chunks in LaTeX documents. Throughout the book I use the older and more established {\emph{Sweave}} style syntax. \\[0.25cm]

\noindent {\bf{Sweave-style}} \\[0.25cm]

Traditional Sweave-style code chunks begin with the following head: \texttt{\textless\textless \textgreater\textgreater=}. The code chunk is closed with an at sign (@).

\begin{knitrout}
    \definecolor{shadecolor}{rgb}{0.969, 0.969, 0.969}
    \color{fgcolor}\begin{kframe}
        \begin{alltt}
\textless\textless \textgreater\textgreater=
\hlcomment{# Example of a Sweave-style code chunk}
StringNumObject <- \hlfunctioncall{cbind}(CharacterVect, NumericVect)
@
        \end{alltt}
    \end{kframe}
\end{knitrout}

\noindent {\bf{Rtex-style}} \\[0.25cm]

Sweave-style code chunk syntax is fairly baroque compared to the Rtex-style syntax. To begin a code chunk in an \texttt{Rtex} file simply type double percent signs followed by \texttt{begin.rcode}, i.e. \texttt{\%\% begin.rcode}. To close the chunk you use double percent signs: \texttt{\%\%}. Each line in the code chunk needs to begin with a single percent sign. For example:

\begin{knitrout}
    \definecolor{shadecolor}{rgb}{0.969, 0.969, 0.969}
    \color{fgcolor}
    \begin{kframe}
        \begin{verbatim}
%% begin.rcode
% # Example of a Rtex-style code chunk
% StringNumObject <- cbind(CharacterVect, NumericVect)
%%
            \end{verbatim}
        \end{kframe}
\end{knitrout}

\paragraph{Code chunk labels}

Each chunk has a label. When a code chunk creates a plot or the output is cached\index{knitr option!cache}--stored for future use--{\emph{knitr}} uses the chunk label for the new file's name. If you do not explicitly give the chunk a label it will be assigned one like: \texttt{unnamed-chunk-1}.

To explicitly assign chunk labels in R Markdown documents place the label name inside of the braces after the \texttt{r}. If we wanted to use the label \texttt{ChunkLabel} we would simply type:

\begin{knitrout}
\definecolor{shadecolor}{rgb}{0.969, 0.969, 0.969}\color{fgcolor}\begin{kframe}
\begin{alltt}
```\{r ChunkLabel\}
\hlcomment{# Example chunk label}
```
\end{alltt}
\end{kframe}
\end{knitrout}


\noindent The same general format applies to the two types of LaTeX chunks. In Sweave-style chunks we would type: \texttt{\textless\textless ChunkLabel\textgreater\textgreater=}. In Rtex-style we use: \texttt{\%\% begin.rcode ChunkLabel}.

Try not to use spaces or periods in your label names. Also remember that chunk labels {\emph{must}} be unique.

\paragraph{Code chunk options}

There are many times when we want to change how our code chunks are knitted and presented. Maybe we only want to show the code and not the results or perhaps we don't want to show the code at all but just a figure that it produces. Maybe we want the figure to be formatted on a page in a certain way. To make these changes, and many others we can specify code chunk options\index{knitr!code chunk options}.

Like chunk labels, you specify options in the chunk head. Place them after the chunk label, separated by a comma. Chunk options are written following pretty much the same rules as regular R command arguments. They have a similar \texttt{OPTIONLABEL=VALUE} structure as arguments. The option values must be written in the same way that argument values are. Character strings need to be inside of quotation marks. The logical \texttt{TRUE} and \texttt{FALSE} operators cannot be written ``true" and ``false". For example, imagine we have a Markdown code chunk called \texttt{ChunkLabel}. If we want to run the {\emph{knitr}} code chunk, but not show the code in the final presentation document we can use the option \texttt{echo=FALSE}.\index{knitr option!echo} 

\begin{knitrout}
\definecolor{shadecolor}{rgb}{0.969, 0.969, 0.969}\color{fgcolor}\begin{kframe}
\begin{alltt}
```\{r ChunkLabel, echo=FALSE\}
StringNumObject <- \hlfunctioncall{cbind}(CharacterVect, NumericVect)
```
\end{alltt}
\end{kframe}
\end{knitrout}


\noindent Note that all labels and code chunk options must be on the same line. Options are separated by commas. The syntax for {\emph{knitr}} options is the same regardless of the markup language. Here is the same chunk option in Rtex-style syntax:

\begin{knitrout}
    \definecolor{shadecolor}{rgb}{0.969, 0.969, 0.969}
    \color{fgcolor}
    \begin{kframe}
        \begin{verbatim}
%% begin.rcode ChunkLabel, echo=FALSE
% # Example of a non-evaluated code chunk
% StringNumObject <- cbind(CharacterVect, NumericVect)
%%
            \end{verbatim}
        \end{kframe}
\end{knitrout}

Throughout this book we will look at a number of different code chunk options. All of the chunk options we will use in this book are listed in Table \ref{ChunkOptionsTable}. For the full list of {\emph{knitr}} options see the {\emph{knitr}} chunk options page maintained by {\emph{knitr}}'s creator Yihui Xie: \url{http://yihui.name/knitr/options#package_options}.


\begin{table}
  \caption{A Selection of {\emph{knitr}} Code Chunk Options}
  \begin{center}
  \label{ChunkOptionsTable}
  \begin{tabular}{l c p{5cm}}
    \hline  
    Chunk Option Label & Type & Description \\[0.25cm] \hline\hline
    \texttt{eval} & Logical & Whether or not to run the chunk. \\[0.25cm]
    \texttt{echo} & Logical & Whether or not to include the code in the presentation document. \\[0.25cm]
    \texttt{error} & Logical & Whether or not to include errors. \\[0.25cm]
    \texttt{engine} & Character & Set the programming language for {\emph{knitr}} to evaluate the code chunk with. \\[0.25cm]
    \texttt{fig.align} & Character & Aligns figures (Note: does not work with R Markdown documents) \\[0.25cm]
    \texttt{fig.path} & Character & Sets the directory where figures will be saved. \\[0.25cm]
    \texttt{include} & Logical & When \texttt{include=FALSE} the chunk is evaluated, but the results are not included in the presentation document. \\[0.25cm]
    \texttt{message} & Logical & Whether or not to include message messages. \\[0.25cm]
    \texttt{out.height} & Numeric & Sets figures' heights in the presentation document. \\[0.25cm]
    \texttt{out.width} & Numeric & Sets figures' widths in the presentation document. \\[0.25cm]   
    \texttt{results} & Character & How to include results in the presentation document. \\[0.25cm]
    \texttt{tidy} & Logical & Whether or not to have \emph{knitr} format printed code chunks. \\[0.25cm]
    \texttt{warning} & Logical & Whether or not to include warnings. \\[0.25cm]
    \hline
  \end{tabular}
  \end{center}
  {\scriptsize{These commands are discussed in more detail in Chapter \ref{StatsModel}.}}
\end{table}

\subsection{Global chunk options}

So far we have only looked at how to set local options\index{knitr!local chunk options} in {\emph{knitr}} code chunks, i.e. options for only one specific chunk. If we want an option to apply to all of the chunks in our document we can set global chunk options\index{knitr!global chunk options}. Options are `global' in the sense that they apply to the entire document. Setting global chunk options helps us create documents that are formatted consistently without having to repetitively specify the same option every time we create a new code chunk. For example, in this book I center almost all of the the figures. Instead of using the \verb|fig.align='center'| option in each code chunk that creates a figure, I set the option globally.

To set a global option first create a new code chunk at the beginning of your document\footnote{In Markdown, you can put global chunk options at the very top of the document. In LaTeX they should be placed after the \texttt{\textbackslash{}begin\{document\}} command (see Chapter \ref{LatexChapter} for more information on how LaTeX documents are structured).} You will probably want to set the option {\tt{echo=FALSE}} so that {\emph{knitr}} doesn't echo the code. Inside the code chunk use {\tt{opts\_chunk\$set}}. You can set any chunk option as an argument to {\tt{opts\_chunk\$set}}. The option will be applied across your document, unless you set a different local option. 

Here is an example of how you can center align all of the figures in a Markdown document created {\emph{knitr}} code chunks. Place the following code at the beginning of the document:

\begin{knitrout}
    \definecolor{shadecolor}{rgb}{0.969, 0.969, 0.969}
    \color{fgcolor}
    \begin{kframe}
        \begin{alltt}
\textless{\textless}ChunkLabel, echo=FALSE\textgreater{\textgreater}=
# Center align all knitr figures
opts_chunk$set(fig.align='center')
@
        \end{alltt}
    \end{kframe}
\end{knitrout}

\noindent W

\subsection{knitr package options}

{\emph{Knitr}} package options\index{package options} affect how the package itself runs. For example, the {\tt{progress}} option can be set as either {\tt{TRUE}} or {\tt{FALSE}}\footnote{It's set as {\tt{TRUE}} by default.} depending on whether or not you want a progress bar\index{progress bar} to be displayed when you knit a code chunk.\footnote{The {\emph{knitr}} progress bar looks like this {\tt{|>>>>>>| 100\%}} and indicates how much of a code chunk has been run.} You can use {\tt{base.dir}} to set the directory where you want all of your figures to be saved to (see Chapter \ref{DirectoriesChapter}) or the {\tt{child.path}} option to specify where child documents are located (see Chapter \ref{LargeDocs}).

You set package options in a similar way as global chunk options with {\tt{opts\_knit\$set}}. For example, to turn off the progress bar when knitting documents include this code at the beginning of the document:

\begin{knitrout}
    \definecolor{shadecolor}{rgb}{0.969, 0.969, 0.969}
    \color{fgcolor}
        \begin{kframe}
            \begin{alltt}
\textless{\textless}ChunkLabel, echo=FALSE\textgreater{\textgreater}=
# Turn off knitr progress bar
opts_knit$set(progress=FALSE)
@
            \end{alltt}
        \end{kframe}
\end{knitrout}


\subsection{Hooks}

You can also set hooks\index{knitr!hook}. Hooks come in two types: chunk hooks and output hooks. Chunk hooks\index{chunk hooks} run a function before or after a code chunk. Output hooks\index{knitr!output hooks} change how the raw output is formatted. I don't cover hooks in much detail in this book. For more information on hooks, please see Yihui Xie's webpage: \url{http://yihui.name/knitr/hooks}.

%%%%%%%%%% Knitr & RStudio
\subsection{knitr \& RStudio}

RStudio is highly integrated with {\emph{knitr}} and the markup languages knitr works with. Because of this integration it is easier to create and compile {\emph{knitr}} documents than doing so in plain R. Most of the RStudio/{\emph{knitr}} features are accessed in the Source pane\index{Source pane}. The Source pane's appearance and capabilities change depending on the type of file you have open in it. RStudio uses a file's extension\index{file extension} to determine what type of file you have open.\footnote{You can manually set how you want the Source pane to act by selecting the file type using the drop down menu in the lower right-hand corner of the Source pane.} We have already seen some of the features the Source pane has for R source code files. Let's now look at how to use {\emph{knitr}} with R source code files as well as the markup formats we cover in this book: R Markdown\index{R Markdown}, and R LaTeX\index{R LaTeX}. \\[0.25cm]

\paragraph{Compiling R source code notebooks}

If you want a quick well formatted account of the code that you ran and the results that you got you can use RStudio's ``Compile Notebook"\index{RStudio!Notebook} capabilities. RStudio uses {\emph{knitr}} to create a standalone HTML file that includes all of the code from an R source file as well as the output. This can be useful for recording the steps you took to do an analysis. You can see an example RStudio Notebook in Figure \ref{NotebookExample}. 

If you want to create a Notebook from an open R source code file simply click the \texttt{Compile Notebook} icon in the Source pane's top bar (see Figure \ref{SourcePanes}).\footnote{Alternatively, \texttt{File} \textrightarrow \; \texttt{Compile Notebook\ldots}} Then click the \texttt{Compile} button in the window that pops up. In Figure \ref{NotebookExample} you can see near the top center right a small globe icon next to the word ``Publish". Clicking this allows you to publish your Notebook to RPubs (\url{http://www.rpubs.com/}). RPubs is a site for sharing your Notebooks over the internet. You can publish not only Notebooks, but also any {\emph{knitr}} Markdown document you compile in RStudio.\label{PublishRPubs}

\begin{figure}
    \caption{RStudio Notebook Example}
    \label{NotebookExample}
    \begin{center}
    
\includegraphics[scale=0.4]{Children/Chapter3/images3/NotebookExample.png}
    \end{center}
\end{figure}

\paragraph{R Markdown} The second image in figure \ref{SourcePanes} is what the Source pane's top bar looks like when you have an R Markdown file open. You'll notice the familiar \texttt{Run} button for running R code. At the far right you can see a new \texttt{Chunks} drop down menu. In this menu you can select \texttt{Insert Chunk} to insert the basic syntax required for a code chunk. There is also an option to \texttt{Run Current Chunk}--i.e. the chunk where your cursor is located--\texttt{Run Next Chunk}, and \texttt{Run All} chunks. You can navigate to a specific chunk using a drop down menu on the bottom left-hand side of the Source pane (not shown). This can be very useful if you are working with a long document. To knit your file click the \texttt{Knit HTML} icon on the left side of the Source pane's top bar. This will create a knitted HTML file as well as a regular Markdown file with highlighted code, output, and figures in your R Markdown's directory. Other useful buttons in the R Markdown Source pane's top bar include the \texttt{ABC} spell check icon and \texttt{MD} icon, which gives you a Markdown syntax reference file in the Help pane.

Another useful RStudio {\emph{knitr}} integration feature is that RStudio can properly highlight both the markup language syntax and the R code in the Source pane. This makes your source code much easier to read and navigate. RStudio can also fold code chunks. This makes navigating through long documents, with long code chunks, much easier. In the first image in Figure \ref{CodeFold} you can see a small downward facing arrow at line 25. If you click this arrow the code chunk will collapse, like in the second image in Figure \ref{CodeFold}. To unfold the chunk, just click on the arrow again.

You may also notice that there are code folding arrows on lines 27 and 34 in the first image. These allow us to fold parts of the code chunk. To enable this option create a comment line with at least one hash before the comment text and at least four after it like this:

\begin{knitrout}
\definecolor{shadecolor}{rgb}{0.969, 0.969, 0.969}\color{fgcolor}\begin{kframe}
\begin{alltt}
\hlcomment{#### An RStudio Foldable Comment ####}
\end{alltt}
\end{kframe}
\end{knitrout}


\noindent You will be able to fold all of the text after this comment up until the next similarly formatted comment (or the end of the chunk).

\begin{figure}[ht!]
    \caption{Folding Code Chunks in RStudio}
    \label{CodeFold}
    \setlength{\belowcaptionskip}{5pt}
    \centering    
    \begin{subfigure}  
        \caption{Not Folded}   
        \includegraphics[width = \textwidth]{Children/Chapter3/images3/MarkdownNoCollapse.png}
    \end{subfigure} \\[0.5cm]
    
    \begin{subfigure}
        \caption{Folded}
        \includegraphics[width = \textwidth]{Children/Chapter3/images3/MarkdownCollapse.png}        
    \end{subfigure}
\end{figure}

\paragraph{R LaTeX}

You can see in the final image in Figure \ref{SourcePanes} that many of the Source pane options for R LaTeX files are the same as R Markdown files. The key differences being that there is a \texttt{Compile PDF} icon instead of \texttt{Knit HTML}. Clicking this icon knits the file and creates a PDF file in your R LaTeX file's directory. There is also a \texttt{Format} icon instead of \texttt{MD}. This actually inserts LaTeX formatting commands into your document for things such as section headings and bullet lists. These commands can be very tedious to type out by hand.


\paragraph{Change default .Rnw knitter}

By default RStudio is set up to use Sweave\index{Sweave} for compiling LaTeX documents. To use {\emph{knitr}} instead of Sweave to knit \texttt{.Rnw} files you should click on \texttt{Tools} in the RStudio menu bar then click on \texttt{Options} window\index{RStudio!Options window}. Once the {\bf{Options}} window opens, click on the \texttt{Sweave} button. Select \texttt{knitr} from the drop down menu for ``Weave files using:". Finally, click \texttt{Apply}.\footnote{In the Mac version of RStudio, you can also access the \texttt{Options} window via \texttt{RStudio} \textrightarrow \texttt{Preferences} in the menu bar.} 

\subsection{knitr \& R}

As {\emph{knitr}} is a regular R package, you can of course knit documents in R (or using the console in RStudio). All of the {\emph{knitr}} syntax in your markup document is the same as before, but instead of clicking a {\tt{Compile PDF}} or {\tt{knit HTML}} button use the {\tt{knit}} command. To knit an example Markdown file {\emph{Example.Rmd}} you first set us the \texttt{setwd} command to set the working directory (for more details see Chapter \ref{DirectoriesChapter}) to the the folder where the {\emph{Example.Rmd}} file is located. In this example it is located on the desktop.\footnote{Using the directory name {\tt{$\sim$/Documents/}} is for Mac computers. Please use alternative syntax discussed in Chapter \ref{DirectoriesChapter} on other types of systems.}

\begin{knitrout}
\definecolor{shadecolor}{rgb}{0.969, 0.969, 0.969}\color{fgcolor}\begin{kframe}
\begin{alltt}
\hlfunctioncall{setwd}(\hlstring{"~/Documents/"})
\end{alltt}
\end{kframe}
\end{knitrout}


\noindent Then you knit the file:

\begin{knitrout}
\definecolor{shadecolor}{rgb}{0.969, 0.969, 0.969}\color{fgcolor}\begin{kframe}
\begin{alltt}
\hlfunctioncall{knit}(input = \hlstring{"Example.Rmd"}, output = \hlstring{"Example.md"})
\end{alltt}
\end{kframe}
\end{knitrout}


\noindent You use the same steps for all other knitable document types. Note that if you do not specify the output file, {\emph{knitr}} will determine what the file name and extension should be. In this example it would come up with the same name and location as you gave it.

In this example, using the {\emph{knit}} command only creates a Markdown file and not an HTML file, as clicking the RStudio {\tt{knit HTML}} did. Likewise, if you use {\tt{knit}} on a {\tt{.Rnw}} file you will only end up with a basic LaTeX {\tt{.tex}} file and not a compiled PDF. To convert the Markdown file into HTML you need to further run the {\tt{.md}} file through the {\tt{markdownToHTML}} command from the {\emph{markdown}} package, i.e.

\begin{knitrout}
\definecolor{shadecolor}{rgb}{0.969, 0.969, 0.969}\color{fgcolor}\begin{kframe}
\begin{alltt}
\hlfunctioncall{mardownToHTML}(file = \hlstring{"Example.md"}, output = \hlstring{"Example.html"})
\end{alltt}
\end{kframe}
\end{knitrout}


\noindent This is a bit tedious. Luckily, there is a command in the {\emph{knitr}} package that combines \texttt{markdownToHTML} and \texttt{knit}. It is called \texttt{knit2html}. You use it like this:

\begin{knitrout}
\definecolor{shadecolor}{rgb}{0.969, 0.969, 0.969}\color{fgcolor}\begin{kframe}
\begin{alltt}
\hlfunctioncall{knit2html}(file = \hlstring{"Example.Rmd"}, output = \hlstring{"Example.html"})
\end{alltt}
\end{kframe}
\end{knitrout}


\noindent If we want to compile a {\tt{.tex}} file in R we run it through the {\tt{texi2pdf}} command in the {\emph{tools}} package. This package will run both LaTeX and BibTeX to create a PDF with a bibliography (see Chapter \ref{LatexChapter} for more details on using BibTeX for bibliographies). Here is a {\tt{texi2pdf}} example:

\begin{knitrout}
\definecolor{shadecolor}{rgb}{0.969, 0.969, 0.969}\color{fgcolor}\begin{kframe}
\begin{alltt}
\hlcomment{# Load tools package}
\hlfunctioncall{library}(tools)

\hlcomment{# Compile pdf}
\hlfunctioncall{texi2pdf}(file = \hlstring{"Example.tex"})
\end{alltt}
\end{kframe}
\end{knitrout}


\noindent Just like with \texttt{knit2html}, you can simplify this process by using the \texttt{knit2pdf} command to compile a PDF file from a \texttt{.Rnw} or \texttt{.Rtex} document.

\section*{Appendix: knitr and Lyx}

You may be more comfortable using a what-you-see-is-what-you-get\index{WYSIWYG} editor, similar to Microsoft Word. Lyx\index{Lyx} is a WYSIWYG LaTeX editor that can be used with \emph{knitr}. I don't cover Lyx in detail in this book, but here is a little information to get you started. 

\paragraph{Set Up}

To set up Lyx so that it can compile \texttt{.Rnw} files click \texttt{Document} in the menu bar then \texttt{Settings}. In the left-hand panel the second option is \texttt{Modules}. Click on \texttt{Modules} and select \texttt{Rnw (knitr)}. Click \texttt{Add} then \texttt{Ok}. Now, compile your LaTeX document in the normal Lyx way.

\paragraph{Code Chunks}

Enter code chunks into TeX Code blocks within your Lyx documents. To create a new TeX Code block select \texttt{Insert} \textrightarrow \texttt{TeX Code}.


% Chapter Chapter 4 For Reproducible Research in R and RStudio
% Christopher Gandrud
% Created: 16/07/2012 05:45:03 pm CEST
% Updated: 10 January 2012




\chapter{Getting Started with File Management}\label{DirectoriesChapter}

Careful file management is crucial for reproducible research. Remember two of the guidelines from Chapter \ref{GettingStartedRR}:

\begin{itemize}
    \item Explicitly tie your files together,
    \item Have a plan to organize, store and make your files available. 
\end{itemize}

\noindent Apart from the times when you have an email exchange (or even meet in person) with someone interested in reproducing your research, the main information independent researchers have about the procedures you used will be stored across many files: data files, analysis files, and presentation files. If these files are well organized and the links tying them together are clear, replication will be much easier. File management is also important for you as a researcher, because if your files are well organized you will be able to more easily make changes, benefit from work you have already done and collaborate with others. 

Using tools such as R, {\emph{knitr}}, and markup languages like LaTeX requires fairly detailed knowledge of where files are stored in your computer. Handling files reproducibly may require you to use command line tools to access and organize your files. R and Unix-like shell programs\index{Unix-like shell program} allow you to control files--creating, deleting, relocating--in powerful and really reproducible ways. By typing these commands you are documenting every step you took. This is a major advantage over graphical user interface-type systems where you organize files by clicking and dragging them with the cursor. However, text commands require you to know your files' specific addresses--their file paths. 

In this chapter we discuss how a reproducible research project may be organized and cover the basics of file path naming conventions\index{file path naming conventions} in Unix, Mac, and Windows systems. We then learn how to organize them with RStudio Projects\index{RStudio!Projects}. Finally, we will cover some basic R and Unix-like shell commands for manipulating files as well as how to navigate through files in RStudio in the {\bf{Files}} pane. The skills you will learn in this chapter will be heavily used in the next chapter (Chapter \ref{Storing}) and throughout the book.

In this chapter we work with locally stored files\index{locally stored}, i.e. files stored on your computer. In the next chapter we will discuss various ways to store and access files remotely stored in the cloud.\index{remotely stored}\index{cloud storage}


\section{File paths \& naming conventions}

All of the operating systems\index{operating systems} covered in this book organize files in hierarchical directories\index{directories}, also know as file trees. To a large extent, directories can be thought of as the folders you usually see on your Windows or Mac desktop.\footnote{To simplify things, I use the terms `directory' and `folder' interchangeably in this book.} They are called `hierarchical' because directories are located inside of other directories, as in Figure \ref{ExampleTree}. 

\subsection{Root directories}

A root directory\index{root directory} is the first level in a disk, such as a hard drive. It is the root out of which the file tree `grows'. All other directories are subdirectories\index{subdirectory} of the root directory.

On Windows computers you can have multiple root directories, one for each storage device or partition of a storage device. The root directory is given a drive letter assignment\index{drive letter assignment}. If you use Windows regularly you will most likely be familiar with the \texttt{C:\textbackslash{}} used to denote the C partition of the the hardrive. This is a root directory. On Unix-like systems, including Macs, the root directory is simply denoted by a forward slash (\texttt{/}) with nothing before it.

\subsection{Subdirectories \& parent directories}

You will probably not store all of your files in the root directory. This would get very messy. Instead you will likely store your files in subdirectories of the root directory. Inside of these subdirectories may be further subdirectories and so on.  Directories inside of other directories are child directories\index{child directory} or subdirectories of a parent directory\index{parent directory}.

On Windows computers separate subdirectories are indicated with a back slash (\textbackslash{}). For example if we have a folder called {\emph{Data}} inside of a folder called {\emph{ExampleProject}} which is located in the C root directory it has the address \texttt{C:\textbackslash{}ExampleProject\textbackslash{}Data}.\footnote{For more information on Windows file path names see this helpful website: \url{http://msdn.microsoft.com/en-us/library/windows/desktop/aa365247(v=vs.85).aspx}} When you type Windows file paths into R you need to use two backslashes rather than one: \texttt{C:\textbackslash{}\textbackslash{}ExampleProject\textbackslash{}\textbackslash{}Data}. This is because the \texttt{\textbackslash{}} is an escape character\index{escape character} in R.\footnote{As we will see in Part IV, it is also a LaTeX escape character.} Escape characters tell R to interpret the next character or sequence of characters differently. For example, on page \pageref{TSVEscape} you'll see how \texttt{\textbackslash{}t} can be interpreted by R as a tab rather than the letter ``t''. Add another escape character to neutralize the escape character so that R interprets it as a backslash. In other words use an escape character to the escape character. Another option for writing Windows file names in R is to use one forward slash (\texttt{/}).  

On Unix-like systems, including Mac computers, directories are indicated with a forward slash (\texttt{/}). The file path of the {\emph{Data}} file on a Unix-like system would be: \texttt{/ExampleProject/Data}

In the book I switch between the two file system naming conventions to expose you to both.

\subsection{Spaces in directory \& file names}

It is generally good practice to avoid putting spaces in your file and directory names. For example, I called the example project parent directory ``ExampleProject" rather than ``Example Project". Spaces in file and directory names can sometimes create problems for compter programs trying to read the file path. It may believe that the space indicates that the path name has ended. To make multi-word names easily readable without using spaces, adopt a convention such as CamelBack\index{CamelBack}. In CamelBack new words are indicated with capital letters, while all other letters are lower case. For example, ``ExampleProject".

\subsection{Working directories}

When you use R and markup languages it is important to keep in mind what your current working directory is\index{working directory}. The working directory is the directory where the program automatically looks for files and other directories, unless you tell it to look elsewhere. It is also where it will save files. Later in this chapter we will cover commands for finding and changing the working directory.

\section{Organizing your research project}

Figure \ref{ExampleTree} gives an example of how the files in a simple reproducible research project could be organized. The project's main parent directory is called {\emph{ExampleProject}}. Inside this directory are three subdirectories: a data gathering directory, an analysis directory, and a presentation directory. Each of these directories contains further subdirectories and files. The {\emph{Presentation}} directory for example contains subdirectories for files that present the findings in article, slideshow, and website formats.

\clearpage
\thispagestyle{plain}
\begin{landscape}
\begin{figure}[th!]
    \caption{Example Research Project File Tree}
    \label{ExampleTree}
    \begin{center}
    
    \documentclass{article}

\usepackage{tikz}

\begin{document}

%%%%%% Example 

\tikzstyle{DirectoryBox} = [draw=rectangle, thick]
  
\begin{tikzpicture}

\end{tikzpicture}



\end{document}
    \end{center}
\end{figure}
\end{landscape}

\begin{wrapfigure}{r}{0.4\textwidth}
    \caption{An Example RStudio Project Menu}
    \label{ProjectMenu}
    \begin{center}
    \includegraphics[width=0.3\textwidth]{Children/Chapter4/images4/ProjectMenu.png}
    \end{center}
\end{wrapfigure}


In addition to the main subdirectories of {\emph{ExampleProject}} you will probably notice a file called {\emph{README.md}} The {\emph{README.md}} file\index{README file} gives an overview of all the files in the project. It should briefly describe the project including things like its title, author(s), topic, any copyright information, and so on. It should also indicate how the folders in the project are organized and give instructions for how to reproduce the project. The README file should be in the main project folder--in our example this is called {\emph{ExampleProject}}--so that it is easy to find. If you are storing your project as a GitHub\index{GitHub} repository (see Chapter \ref{Storing}) and the file is called \texttt{README} its contents will automatically be displayed on the repository's main page. If the \texttt{README} file is written using Markdown, it will also be properly formatted. Figure \ref{BookRepository} shows an example of this.

It is good practice to dynamically include the system information for the R session you used to create the project. To do this you can write your README file with R Markdown (see Chapter \ref{LargeDocs}). Simply include the \texttt{sessionInfo()} command in a code chunk in the R Markdown document. If you knit this file immediately after knitting your presentation document it will record the information for that session.

You can also dynamically include session info in a LaTeX document. To do this use the {\tt{toLatex}} command in a code chunk. The code chunk should have the option \verb|results='asis'|. The code is:

\begin{knitrout}
\definecolor{shadecolor}{rgb}{0.969, 0.969, 0.969}\color{fgcolor}\begin{kframe}
\begin{alltt}
\hlfunctioncall{toLatex}(\hlfunctioncall{sessionInfo}())
\end{alltt}
\end{kframe}
\end{knitrout}


%% Make file discussion?

\section{Setting directories as RStudio Projects}

If you are using RStudio, you may want to organize your files as Projects\index{RStudio!Projects}.\label{CreateRStudioProject} You can turn a normal directory into an RStudio Project by clicking on \texttt{Project} in the RStudio menu bar and selecting \texttt{Create Project\ldots}. A new window will pop up. Select the option \texttt{Existing Directory}. Find the directory you want to turn into an RStudio Project by clicking on the \texttt{Browse} button. Finally, select \texttt{Create Project}. You will also notice in the Create Project pop up window that you can build new project directories and create a project from a directory already under version control\index{version control} (we'll do this at the end of Chapter \ref{Storing}). When you create a new project you will see that RStudio has put a file with the extension \texttt{.Rproj} into the directory.

Making your research project directories RStudio Projects is useful for a number of reasons:

\begin{itemize}
    \item the project is listed in RStudio's Project menu where it can be opened easily (see Figure \ref{ProjectMenu}).
    \item when you open the \texttt{.Rproj} file RStudio automatically sets the working directory to the project's directory and loads the workspace, history, and source code files you were last working on.
    \item you can set project specific options like whether PDF presentation documents should be compiled with Sweave or {\emph{knitr}}.
    \item when you close the project your R workspace and history are saved in the project directory,
    \item it helps you version control your files,
    \item you can build your Project--run the files in a specific way--with Makefiles.
\end{itemize}

%%%%%%%%%%%%%%% File Manipulation
\section{R file manipulation commands}

R has a range of commands for handling and navigating through files. Including these commands in your source code files allows you to more easily replicate your actions.

\paragraph{{\tt{getwd}}}\index{R command!getwd}

To find your current working directory use the \texttt{getwed} command:

\begin{knitrout}
\definecolor{shadecolor}{rgb}{0.969, 0.969, 0.969}\color{fgcolor}\begin{kframe}
\begin{alltt}
\hlfunctioncall{getwd}()
\end{alltt}
\begin{verbatim}
## [1] "/git_repositories/Rep-Res-Book/Source/Children/Chapter4"
\end{verbatim}
\end{kframe}
\end{knitrout}


\noindent The example here shows you the current working directory that was used while knitting this chapter.

\paragraph{{\tt{list.files}}}\index{R command!list.files}

Use the \texttt{list.files} command to see all of the files and subdirectories in the current working directory. You can list the files in other directories too by adding the directory path as an argument to the command. 

\begin{knitrout}
\definecolor{shadecolor}{rgb}{0.969, 0.969, 0.969}\color{fgcolor}\begin{kframe}
\begin{alltt}
\hlfunctioncall{list.files}()
\end{alltt}
\begin{verbatim}
## [1] "chapter4.Rnw" "images4"
\end{verbatim}
\end{kframe}
\end{knitrout}


\noindent You can see that the \emph{Chapter4} folder has the file \emph{chapter4.Rnw} (the markup file used to create this chapter) and a child directory called \emph{images4} where I stored the original versions of the figures show in this chapter.

\paragraph{{\tt{setwd}}}\index{R command!setwd}

The {\tt{setwd}} command sets the current working directory\index{working directory}. For example, if we are on a Mac or other Unix-like computer we can set the working directory to the {\emph{GatherSource}} directory in our Example Project (see Figure \ref{ExampleTree}) like this

\begin{knitrout}
\definecolor{shadecolor}{rgb}{0.969, 0.969, 0.969}\color{fgcolor}\begin{kframe}
\begin{alltt}
\hlfunctioncall{setwd}(\hlstring{"/ExampleProject/Data/GatherSource"})
\end{alltt}
\end{kframe}
\end{knitrout}


\noindent Now R will automatically look in the {\emph{GatherSource}} folder for files and will save new files into this folder, unless we explicitly tell it to do otherwise.

\paragraph{{\tt{dir.create}}}\index{R command!dir.create}

Sometimes you may want to create a new directory. You can use the {\tt{dir.create}} command to do this.\footnote{Note: you will need the correct system permissions to be able to do this.} For example to create a {\emph{ExampleProject}} file in the root C directory on a Windows computer type:

\begin{knitrout}
\definecolor{shadecolor}{rgb}{0.969, 0.969, 0.969}\color{fgcolor}\begin{kframe}
\begin{alltt}
\hlfunctioncall{dir.create}(\hlstring{"C:\textbackslash{}\textbackslash{}ExampleProject"})
\end{alltt}
\end{kframe}
\end{knitrout}


\paragraph{{\tt{file.create}}}\index{R command!file.create}

Similarly, you can create a new blank file with the \texttt{file.create} command. To add a blank R source code file called {\emph{SourceCode.R}} to the {\emph{ExampleProject}} directory on the C drive use:

\begin{knitrout}
\definecolor{shadecolor}{rgb}{0.969, 0.969, 0.969}\color{fgcolor}\begin{kframe}
\begin{alltt}
\hlfunctioncall{file.create}(\hlstring{"C:\textbackslash{}\textbackslash{}ExampleProject\textbackslash{}\textbackslash{}SourceCode.R"})
\end{alltt}
\end{kframe}
\end{knitrout}


\paragraph{{\tt{cat}}}\index{R command!cat}\label{catR}

If you want to create a new file and put text into it use the \texttt{cat} (concatenate and print). For example to create a new file in the current working directory called \emph{ExampleEcho.md} that includes the text ``Reproducible Research with R and RStudio'' type:

\begin{knitrout}
\definecolor{shadecolor}{rgb}{0.969, 0.969, 0.969}\color{fgcolor}\begin{kframe}
\begin{alltt}
\hlfunctioncall{cat}(\hlstring{"Reproducible Research with R and RStudio"},
    file = \hlstring{"ExampleCat.md"})
\end{alltt}
\end{kframe}
\end{knitrout}


\noindent You can use \texttt{cat} to also print the contents of a one or more objects to a file. In this example we created a Markdown formatted file by using the \texttt{.md} file extension. We could of course change the file extension to \texttt{.R} to set it as an R source code file, \texttt{.Rnw} to create a \emph{knitr} LaTeX file and so on.

The \texttt{cat} command will overwrite existing files with the new contents. To add the text to existing files use the \texttt{append = TRUE} argument.

\begin{knitrout}
\definecolor{shadecolor}{rgb}{0.969, 0.969, 0.969}\color{fgcolor}\begin{kframe}
\begin{alltt}
\hlfunctioncall{cat}(\hlstring{"More Text"}, file = \hlstring{"ExampleCat.md"},
    append = TRUE)
\end{alltt}
\end{kframe}
\end{knitrout}


\paragraph{{\tt{unlink}}}\index{R command!unlink}

Finally, you can use the {\tt{unlink}} command to delete files and directories. 

\begin{knitrout}
\definecolor{shadecolor}{rgb}{0.969, 0.969, 0.969}\color{fgcolor}\begin{kframe}
\begin{alltt}
\hlfunctioncall{unlink}(\hlstring{"C:\textbackslash{}\textbackslash{}ExampleProject\textbackslash{}\textbackslash{}SourceCode.R"})
\end{alltt}
\end{kframe}
\end{knitrout}


\noindent \textbf{Important} the \texttt{unlink} command permanently deletes files, so be very careful using this command.

\paragraph{{\tt{file.rename}}}\index{R command!file.rename}

You can use the \texttt{file.rename} to obviously rename a file. It can also be used to move a file from one directory to another. For example, imagine that we want to move the \emph{ExampleCat.md} file from the directory \emph{ExampleProject} to one called \emph{MarkdownFiles} that we already created.\footnote{The \texttt{file.rename} command won't create new directories. To move a file to a new directory you will need to create the directory first with \texttt{dir.create}.}

\begin{knitrout}
\definecolor{shadecolor}{rgb}{0.969, 0.969, 0.969}\color{fgcolor}\begin{kframe}
\begin{alltt}
\hlfunctioncall{file.rename}(from = \hlstring{"C:\textbackslash{}\textbackslash{}ExampleProject\textbackslash{}\textbackslash{}ExampleCat.md"}, 
            to = \hlstring{"C:\textbackslash{}\textbackslash{}MarkdownFiles\textbackslash{}\textbackslash{}ExampleCat.md"})
\end{alltt}
\end{kframe}
\end{knitrout}


\paragraph{{\tt{file.copy}}}\index{R command!file.copy}

The \texttt{file.rename} fully moves a file from one directory to another. To copy the file to another directory use the \texttt{file.rename} command. It has the same syntax as \texttt{file.rename}:

\begin{knitrout}
\definecolor{shadecolor}{rgb}{0.969, 0.969, 0.969}\color{fgcolor}\begin{kframe}
\begin{alltt}
\hlfunctioncall{file.copy}(from = \hlstring{"C:\textbackslash{}\textbackslash{}ExampleProject\textbackslash{}\textbackslash{}ExampleCat.md"}, 
            to = \hlstring{"C:\textbackslash{}\textbackslash{}MarkdownFiles\textbackslash{}\textbackslash{}ExampleCat.md"})
\end{alltt}
\end{kframe}
\end{knitrout}


\section{Unix-like shell commands for file management} 

\begin{wrapfigure}{r}{0.5\textwidth}
    \caption{The RStudio Files Pane} %\\[0.25cm]
    \label{FilesPane}
        \begin{center}    
        \includegraphics[width=0.45\textwidth]{Children/Chapter4/images4/RStudioFiles.png}
        \end{center}
\end{wrapfigure}

Though this book is mostly focused on using R for reproducible research it can be useful to use a Unix-like shell program\index{Unix-like shell program} to manipulate files in large projects. A command line shell program is simply a program in which you to type commands to interact with your computer's operating system. We will especially return to shell commands in the next chapter when we discuss Git\index{Git} version control and makefiles\index{makefile} for collecting data in Chapter \ref{DataGather}, as well as the command line program Pandoc (Chapter \ref{LargeDocs}). We don't have enough space to fully introduce shell programs or even all of the commands for manipulating files. We are just going to cover some of the basic and most useful commands For good introductions for Unix and Mac OS 10 computers see William E. Shotts Jr.'s book on the Linux command-line \cite[]{ShottsJr2012}. For Windows users, Microsoft maintains a tutorial on Windows PowerShell at \url{http://technet.microsoft.com/en-us/library/hh848793}. The commands discussed in this chapter should work in both Unix-like shells and Windows Powershell. That being said, I have not tested all of them in Powershell. 

It's important at this point to highlight a key difference between R and Unix-like shell syntax. In shell commands you don't need to put parentheses around your arguments. For example if I want to change my working directory to my Mac Desktop in a shell using the \texttt{cd} command I simply type:\footnote{Many shell code examples in other sources include the shell prompt, like the \texttt{\$} in Bash. It's like R's \texttt{\textgreater} prompt. I don't include the prompt in code examples in this book because you don't type them.}

\begin{knitrout}
\definecolor{shadecolor}{rgb}{0.969, 0.969, 0.969}\color{fgcolor}\begin{kframe}
\begin{verbatim}
cd /Users/Me/Desktop
\end{verbatim}
\end{kframe}
\end{knitrout}


\noindent In this example \texttt{Me} is my user name. 

\paragraph{{\tt{cd}}}\index{shell command!cd}

As we just saw, to change the working directory in the shell just use the {\tt{cd}} (change directory) command. If you are in a child directory and want to change the working directory to its parent you can simply type:

\begin{knitrout}
\definecolor{shadecolor}{rgb}{0.969, 0.969, 0.969}\color{fgcolor}\begin{kframe}
\begin{verbatim}
cd -
\end{verbatim}
\end{kframe}
\end{knitrout}


\noindent If, for example, our current working directory was \emph{/User/Me/Desktop} and we typed \texttt{cd -} then the working directory would change to \emph{/User/Me}.

\paragraph{{\tt{pwd}}}\index{shell command!pwd}

To find your current working directory use the \texttt{pwd} command (present working directory). This is essentially the same as R's \texttt{getwd} command.

\begin{knitrout}
\definecolor{shadecolor}{rgb}{0.969, 0.969, 0.969}\color{fgcolor}\begin{kframe}
\begin{verbatim}
pwd

## /Users/Me/Desktop
\end{verbatim}
\end{kframe}
\end{knitrout}


\paragraph{{\tt{ls}}}\index{shell command!ls}

The \texttt{ls} (list) command works very similarly to R's \texttt{list.files} command. It shows you what is in the current working directory.

\begin{knitrout}
\definecolor{shadecolor}{rgb}{0.969, 0.969, 0.969}\color{fgcolor}\begin{kframe}
\begin{verbatim}
ls

## chapter4.Rnw images4
\end{verbatim}
\end{kframe}
\end{knitrout}


\paragraph{{\tt{mkdir}}}\index{shell command!mkdir}

Use \texttt{mkdir} to create a new directory. For example, if I wanted to create a directory in my Linux root directory called {\emph{NewDirectory}} I would type:

\begin{knitrout}
\definecolor{shadecolor}{rgb}{0.969, 0.969, 0.969}\color{fgcolor}\begin{kframe}
\begin{verbatim}
mkdir /NewDirectory
\end{verbatim}
\end{kframe}
\end{knitrout}


\noindent If running this code gives you an error message like this:

\begin{knitrout}
\definecolor{shadecolor}{rgb}{0.969, 0.969, 0.969}\color{fgcolor}\begin{kframe}
\begin{verbatim}
mkdir: /NewDirectory: Permission denied
\end{verbatim}
\end{kframe}
\end{knitrout}


\noindent you simply need to use the \texttt{sudo}\index{shell command!sudo} command to run the command with higher privileges. 

\begin{knitrout}
\definecolor{shadecolor}{rgb}{0.969, 0.969, 0.969}\color{fgcolor}\begin{kframe}
\begin{verbatim}
sudo mkdir /NewDirectory
\end{verbatim}
\end{kframe}
\end{knitrout}


\noindent Running this code will prompt you to enter your administrator password.

\paragraph{{\tt{echo}}}\index{shell command!echo}

There are a number of ways to create new files in Unix-like shells. One of the simplest ways is with the \texttt{echo} command. This command simply prints its arguments. For example:

\begin{knitrout}
\definecolor{shadecolor}{rgb}{0.969, 0.969, 0.969}\color{fgcolor}\begin{kframe}
\begin{verbatim}
echo Reproducible Research with R and RStudio
## Reproducible Research with R and RStudio
\end{verbatim}
\end{kframe}
\end{knitrout}


\noindent If you add the greater than symbol (\verb|>|) after the text you want to print then a file name, \texttt{echo} will create the file (if it doesn't already exist) in the current working directory then print the text into the file.

\begin{knitrout}
\definecolor{shadecolor}{rgb}{0.969, 0.969, 0.969}\color{fgcolor}\begin{kframe}
\begin{verbatim}
echo Reproducible Research with R and RStudio > ExampleEcho.md
\end{verbatim}
\end{kframe}
\end{knitrout}


\noindent Using only one greater than sign will completely erase the \emph{ExampleEcho.txt} file's contents and replace them with \texttt{Reproducible Research with R and RStudo}. To add the text at the end of an existing file use two greater than signs (\verb|>>|).

\begin{knitrout}
\definecolor{shadecolor}{rgb}{0.969, 0.969, 0.969}\color{fgcolor}\begin{kframe}
\begin{verbatim}
echo More text. >> ExampleEcho.md
\end{verbatim}
\end{kframe}
\end{knitrout}


\noindent There is also a \texttt{cat} shell command.\index{shell command!cat} It works slightly differently than the R version of the command and I don't cover it here.

\paragraph{{\tt{rm}}}\index{shell command!rm}

The {\tt{rm}} command is similar to R's {\tt{unlink}} command. It removes (deletes) files or directories. Again, be careful when using this command, because it permanently deletes the files or directories.

As we saw in Chapter \ref{GettingStartedRKnitr}, R also has an \texttt{rm} command. It is different because it removes objects from your R workspace rather than files from your working directory.

\begin{knitrout}
\definecolor{shadecolor}{rgb}{0.969, 0.969, 0.969}\color{fgcolor}\begin{kframe}
\begin{verbatim}
rm ExampleEcho.md
\end{verbatim}
\end{kframe}
\end{knitrout}


\paragraph{{\tt{mv}}}\index{shell command!mv}

To move a file from one directory to another with the shell use the \texttt{mv} (move) command. For example, to move the file \emph{ExampleEcho.md} from \emph{ExampleProjects} to \emph{MarkdownFiles} use the following code and imagine both directories are in the root directory:\footnote{If they were not in the root directory we would not place a forward slash at the begining.}

\begin{knitrout}
\definecolor{shadecolor}{rgb}{0.969, 0.969, 0.969}\color{fgcolor}\begin{kframe}
\begin{verbatim}
mv /ExampleProject/ExampleEcho.md /MarkdownFiles
\end{verbatim}
\end{kframe}
\end{knitrout}


\noindent Note that like the \emph{MarkdownFiles} directory must already exist. You can also use the \texttt{mv} command to simply rename files, just like the R command \texttt{file.rename}.

\paragraph{{\tt{cp}}}\index{shell command!cp}

The \texttt{mv} command completely moves a file from one directory to another. To simply copy a version of the file to a new directory use the \texttt{cp} command. The syntax is similar to \texttt{mv}:

\begin{knitrout}
\definecolor{shadecolor}{rgb}{0.969, 0.969, 0.969}\color{fgcolor}\begin{kframe}
\begin{verbatim}
cp /ExampleProject/ExampleEcho.md /MarkdownFiles
\end{verbatim}
\end{kframe}
\end{knitrout}


\paragraph{{\tt{system}} (R command)}\index{R command!system}\label{systemRcommand}

You can run shell commands from within R using R's \texttt{system} command. For example, to run the \texttt{echo} command from within R type:

\begin{knitrout}
\definecolor{shadecolor}{rgb}{0.969, 0.969, 0.969}\color{fgcolor}\begin{kframe}
\begin{alltt}
\hlfunctioncall{system}(\hlstring{"echo Text to Add > ExampleEcho.md"})
\end{alltt}
\end{kframe}
\end{knitrout}



\section{File navigation in RStudio}

The RStudio \textbf{Files} pane allows us to navigate and do some basic file manipulation. Figure \ref{FilesPane} shows us what this pane looks like. This pane allows us to navigate to specific files and folder, delete and rename files. To select a folder as the working directory tick the dialog box next to the file then click the \texttt{More} button and select \texttt{Set As Working Directory}. Under the \texttt{More} button you will also find options to \texttt{Move} and \texttt{Copy} files.

The \textbf{Files} pane is a GUI, so our actions in the \textbf{Files} pane are not as easily reproducible as the commands we learned earlier in this chapter. 



% Part 2, include child documents
\part{Data Gathering and Storage}







