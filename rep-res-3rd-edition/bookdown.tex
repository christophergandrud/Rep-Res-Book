\PassOptionsToPackage{unicode=true}{hyperref} % options for packages loaded elsewhere
\PassOptionsToPackage{hyphens}{url}
\PassOptionsToPackage{dvipsnames,svgnames*,x11names*}{xcolor}
%
\documentclass[]{krantz}
\usepackage{lmodern}
\usepackage{amssymb,amsmath}
\usepackage{ifxetex,ifluatex}
\usepackage{fixltx2e} % provides \textsubscript
\ifnum 0\ifxetex 1\fi\ifluatex 1\fi=0 % if pdftex
  \usepackage[T1]{fontenc}
  \usepackage[utf8]{inputenc}
  \usepackage{textcomp} % provides euro and other symbols
\else % if luatex or xelatex
  \usepackage{unicode-math}
  \defaultfontfeatures{Ligatures=TeX,Scale=MatchLowercase}
\fi
% use upquote if available, for straight quotes in verbatim environments
\IfFileExists{upquote.sty}{\usepackage{upquote}}{}
% use microtype if available
\IfFileExists{microtype.sty}{%
\usepackage[]{microtype}
\UseMicrotypeSet[protrusion]{basicmath} % disable protrusion for tt fonts
}{}
\IfFileExists{parskip.sty}{%
\usepackage{parskip}
}{% else
\setlength{\parindent}{0pt}
\setlength{\parskip}{6pt plus 2pt minus 1pt}
}
\usepackage{xcolor}
\usepackage{hyperref}
\hypersetup{
            pdftitle={Reproducible Research with R and RStudio (Third Edition)},
            pdfauthor={Christopher Gandrud},
            colorlinks=true,
            linkcolor=Maroon,
            citecolor=Blue,
            urlcolor=Blue,
            breaklinks=true}
\urlstyle{same}  % don't use monospace font for urls
\usepackage{color}
\usepackage{fancyvrb}
\newcommand{\VerbBar}{|}
\newcommand{\VERB}{\Verb[commandchars=\\\{\}]}
\DefineVerbatimEnvironment{Highlighting}{Verbatim}{commandchars=\\\{\}}
% Add ',fontsize=\small' for more characters per line
\usepackage{framed}
\definecolor{shadecolor}{RGB}{248,248,248}
\newenvironment{Shaded}{\begin{snugshade}}{\end{snugshade}}
\newcommand{\AlertTok}[1]{\textcolor[rgb]{0.33,0.33,0.33}{#1}}
\newcommand{\AnnotationTok}[1]{\textcolor[rgb]{0.37,0.37,0.37}{\textbf{\textit{#1}}}}
\newcommand{\AttributeTok}[1]{\textcolor[rgb]{0.61,0.61,0.61}{#1}}
\newcommand{\BaseNTok}[1]{\textcolor[rgb]{0.06,0.06,0.06}{#1}}
\newcommand{\BuiltInTok}[1]{#1}
\newcommand{\CharTok}[1]{\textcolor[rgb]{0.5,0.5,0.5}{#1}}
\newcommand{\CommentTok}[1]{\textcolor[rgb]{0.37,0.37,0.37}{\textit{#1}}}
\newcommand{\CommentVarTok}[1]{\textcolor[rgb]{0.37,0.37,0.37}{\textbf{\textit{#1}}}}
\newcommand{\ConstantTok}[1]{\textcolor[rgb]{0,0,0}{#1}}
\newcommand{\ControlFlowTok}[1]{\textcolor[rgb]{0.27,0.27,0.27}{\textbf{#1}}}
\newcommand{\DataTypeTok}[1]{\textcolor[rgb]{0.27,0.27,0.27}{#1}}
\newcommand{\DecValTok}[1]{\textcolor[rgb]{0.06,0.06,0.06}{#1}}
\newcommand{\DocumentationTok}[1]{\textcolor[rgb]{0.37,0.37,0.37}{\textbf{\textit{#1}}}}
\newcommand{\ErrorTok}[1]{\textcolor[rgb]{0.14,0.14,0.14}{\textbf{#1}}}
\newcommand{\ExtensionTok}[1]{#1}
\newcommand{\FloatTok}[1]{\textcolor[rgb]{0.06,0.06,0.06}{#1}}
\newcommand{\FunctionTok}[1]{\textcolor[rgb]{0,0,0}{#1}}
\newcommand{\ImportTok}[1]{#1}
\newcommand{\InformationTok}[1]{\textcolor[rgb]{0.37,0.37,0.37}{\textbf{\textit{#1}}}}
\newcommand{\KeywordTok}[1]{\textcolor[rgb]{0.27,0.27,0.27}{\textbf{#1}}}
\newcommand{\NormalTok}[1]{#1}
\newcommand{\OperatorTok}[1]{\textcolor[rgb]{0.43,0.43,0.43}{\textbf{#1}}}
\newcommand{\OtherTok}[1]{\textcolor[rgb]{0.37,0.37,0.37}{#1}}
\newcommand{\PreprocessorTok}[1]{\textcolor[rgb]{0.37,0.37,0.37}{\textit{#1}}}
\newcommand{\RegionMarkerTok}[1]{#1}
\newcommand{\SpecialCharTok}[1]{\textcolor[rgb]{0,0,0}{#1}}
\newcommand{\SpecialStringTok}[1]{\textcolor[rgb]{0.5,0.5,0.5}{#1}}
\newcommand{\StringTok}[1]{\textcolor[rgb]{0.5,0.5,0.5}{#1}}
\newcommand{\VariableTok}[1]{\textcolor[rgb]{0,0,0}{#1}}
\newcommand{\VerbatimStringTok}[1]{\textcolor[rgb]{0.5,0.5,0.5}{#1}}
\newcommand{\WarningTok}[1]{\textcolor[rgb]{0.37,0.37,0.37}{\textbf{\textit{#1}}}}
\usepackage{longtable,booktabs}
% Fix footnotes in tables (requires footnote package)
\IfFileExists{footnote.sty}{\usepackage{footnote}\makesavenoteenv{longtable}}{}
\usepackage{graphicx,grffile}
\makeatletter
\def\maxwidth{\ifdim\Gin@nat@width>\linewidth\linewidth\else\Gin@nat@width\fi}
\def\maxheight{\ifdim\Gin@nat@height>\textheight\textheight\else\Gin@nat@height\fi}
\makeatother
% Scale images if necessary, so that they will not overflow the page
% margins by default, and it is still possible to overwrite the defaults
% using explicit options in \includegraphics[width, height, ...]{}
\setkeys{Gin}{width=\maxwidth,height=\maxheight,keepaspectratio}
\setlength{\emergencystretch}{3em}  % prevent overfull lines
\providecommand{\tightlist}{%
  \setlength{\itemsep}{0pt}\setlength{\parskip}{0pt}}
\setcounter{secnumdepth}{5}
% Redefines (sub)paragraphs to behave more like sections
\ifx\paragraph\undefined\else
\let\oldparagraph\paragraph
\renewcommand{\paragraph}[1]{\oldparagraph{#1}\mbox{}}
\fi
\ifx\subparagraph\undefined\else
\let\oldsubparagraph\subparagraph
\renewcommand{\subparagraph}[1]{\oldsubparagraph{#1}\mbox{}}
\fi

% set default figure placement to htbp
\makeatletter
\def\fps@figure{htbp}
\makeatother

\usepackage{booktabs}
\usepackage{longtable}
\usepackage[bf,singlelinecheck=off]{caption}

\usepackage{framed,color}
\definecolor{shadecolor}{RGB}{248,248,248}

\renewcommand{\textfraction}{0.05}
\renewcommand{\topfraction}{0.8}
\renewcommand{\bottomfraction}{0.8}
\renewcommand{\floatpagefraction}{0.75}

\renewenvironment{quote}{\begin{VF}}{\end{VF}}
\let\oldhref\href
\renewcommand{\href}[2]{#2\footnote{\url{#1}}}

\makeatletter
\newenvironment{kframe}{%
\medskip{}
\setlength{\fboxsep}{.8em}
 \def\at@end@of@kframe{}%
 \ifinner\ifhmode%
  \def\at@end@of@kframe{\end{minipage}}%
  \begin{minipage}{\columnwidth}%
 \fi\fi%
 \def\FrameCommand##1{\hskip\@totalleftmargin \hskip-\fboxsep
 \colorbox{shadecolor}{##1}\hskip-\fboxsep
     % There is no \\@totalrightmargin, so:
     \hskip-\linewidth \hskip-\@totalleftmargin \hskip\columnwidth}%
 \MakeFramed {\advance\hsize-\width
   \@totalleftmargin\z@ \linewidth\hsize
   \@setminipage}}%
 {\par\unskip\endMakeFramed%
 \at@end@of@kframe}
\makeatother

\renewenvironment{Shaded}{\begin{kframe}}{\end{kframe}}

\usepackage{makeidx}
\makeindex

\urlstyle{tt}

\usepackage{amsthm}
\makeatletter
\def\thm@space@setup{%
  \thm@preskip=8pt plus 2pt minus 4pt
  \thm@postskip=\thm@preskip
}
\makeatother

\usepackage{tikzpicture}

\usepackage{tikz}
    \usetikzlibrary{trees}
    \usetikzlibrary{decorations.pathmorphing}
    \usetikzlibrary{shapes,arrows}

\frontmatter
\usepackage[]{natbib}
\bibliographystyle{apalike}

\title{Reproducible Research with R and RStudio (Third Edition)}
\author{Christopher Gandrud}
\date{2018-09-23}

\usepackage{amsthm}
\newtheorem{theorem}{Theorem}[chapter]
\newtheorem{lemma}{Lemma}[chapter]
\theoremstyle{definition}
\newtheorem{definition}{Definition}[chapter]
\newtheorem{corollary}{Corollary}[chapter]
\newtheorem{proposition}{Proposition}[chapter]
\theoremstyle{definition}
\newtheorem{example}{Example}[chapter]
\theoremstyle{definition}
\newtheorem{exercise}{Exercise}[chapter]
\theoremstyle{remark}
\newtheorem*{remark}{Remark}
\newtheorem*{solution}{Solution}
\begin{document}
\maketitle

% you may need to leave a few empty pages before the dedication page

%\cleardoublepage\newpage\thispagestyle{empty}\null
%\cleardoublepage\newpage\thispagestyle{empty}\null
%\cleardoublepage\newpage
\thispagestyle{empty}

\begin{center}
To my wife,

who is currently at the movies with our son, so that I can finish the third edition.
%\includegraphics{images/dedication.pdf}
\end{center}

\setlength{\abovedisplayskip}{-5pt}
\setlength{\abovedisplayshortskip}{-5pt}

{
\hypersetup{linkcolor=}
\setcounter{tocdepth}{2}
\tableofcontents
}
\listoftables
\listoffigures
\hypertarget{preface}{%
\chapter*{Preface}\label{preface}}


\hypertarget{my-motivation}{%
\section*{My motivation}\label{my-motivation}}


This book has its genesis in my PhD research at the London School of
Economics. I started the degree with questions about the 2008/09
financial crisis and planned to spend most of my time researching
capital adequacy requirements. But I quickly realized that I would
actually spend a large proportion of my time learning the day-to-day
tasks of data gathering, analysis, and results presentation. After
plodding through for a while with Word, Excel, and Stata, my breaking
point came while reentering results into a regression table after I had
tweaked one of my statistical models, yet again. Surely there was a
better way to \emph{do} research that would allow me to spend more time
answering my research questions. Making research reproducible for others
also means making it better organized and efficient for yourself. My
search for a better way led me straight to the tools for reproducible
computational research.

The reproducible research community is very active, knowledgeable, and
helpful. Nonetheless, I often encountered holes in this collective
knowledge, or at least had no resource organize it all together as a
whole. That is my intention for this book: to bring together the skills
I have picked up for actually doing and presenting computational
research. Hopefully, the book, along with making reproducible research
more widely used, will save researchers hours of googling, so they can
spend more time addressing their research questions.

\hypertarget{changes-to-the-third-edition}{%
\section*{Changes to the Third
Edition}\label{changes-to-the-third-edition}}


\textbf{To WRITE}

\begin{itemize}
\tightlist
\item
  The book is created using \emph{bookdown} \citep{R-bookdown}, a format
  that builds on \emph{rmarkdown} to compile books into many different
  formats.
\end{itemize}

\hypertarget{changes-to-the-second-edition}{%
\section*{Changes to the Second
Edition}\label{changes-to-the-second-edition}}


The tools of reproducible research have developed rapidly since the
first edition of this book was published just two years ago. The second
edition has been updated to incorporate the most important of these
advancements, including discussions of:

\begin{itemize}
\item
  The \emph{rmarkdown} package, which allows you to create reproducible
  research documents in PDF, HTML, and Microsoft Word formats using the
  simple and intuitive Markdown syntax.
\item
  Improvements and changes to RStudio's interface and capabilities, such
  as its new tools for handling R Markdown documents.
\item
  Expanded \emph{knitr} R code chunk capabilities.
\item
  The \texttt{kable()} function in the \emph{knitr} package and the
  \emph{texreg} package for dynamically creating tables to present your
  data and statistical results.
\item
  An improved discussion of file organization allowing you to take full
  advantage of relative file paths so that your documents are more
  easily reproducible across computers and systems.
\item
  The \emph{dplyr}, \emph{magrittr}, and \emph{tidyr} packages for fast
  data manipulation.
\item
  Numerous changes to R syntax in user-created packages.
\item
  Changes to GitHub's and Dropbox's interfaces.
\end{itemize}

\hypertarget{acknowledgments}{%
\section*{Acknowledgments}\label{acknowledgments}}


I would not have been able to write this book without many people's
advice and support. Foremost is John Kimmel, acquisitions editor at
Chapman and Hall. He approached me in Spring 2012 with the general idea
and opportunity for this book. Other editors at Chapman and Hall and
Taylor and Francis have greatly contributed to this project, including
Marcus Fontaine. I would also like to thank all of the book's reviewers
whose helpful comments have greatly improved it. The first edition's
reviewers include:

\begin{itemize}
\tightlist
\item
  Jeromy Anglim, Deakin University
\item
  Karl Broman, University of Wisconsin, Madison
\item
  Jake Bowers, University of Illinois, Urbana-Champaign
\item
  Corey Chivers, McGill University
\item
  Mark M. Fredrickson, University of Illinois, Urbana-Champaign
\item
  Benjamin Lauderdale, London School of Economics
\item
  Ramnath Vaidyanathan, McGill University
\end{itemize}

and there have been many other annonymous reviewers who have provided
great feedback over the years.

The developer and blogging community has also been incredibly important
for making this book possible. Foremost among these people is Yihui Xie.
He is the main developer behind the \emph{knitr} package, co-developer
of \emph{rmarkdown}, and also an avid blog writer and commenter. Without
him the ability to do reproducible research would be much harder and the
blogging community that spreads knowledge about how to do these things
would be poorer. Other great contributors to the reproducible research
community include Carl Boettiger, Karl Broman, Markus Gesmann (who
developed \emph{googleVis}), Rob Hyndman, and Hadley Wickham (who has
developed numerous very useful R packages). Thank you also to Victoria
Stodden and Michael Malecki for helpful suggestions. And, of course,
thank you to everyone at RStudio (especially JJ Allaire) for creating an
increasingly useful program for reproducible research.

The second edition has benefited immensely from first edition readers'
comments and suggestions. For a list of their valuable contributions,
please see the book's GitHub Issues page
\url{https://GitHub.com/christophergandrud/Rep-Res-Book/issues} and the
first edition's Errata page
\url{http://christophergandrud.GitHub.io/RepResR-RStudio/errata.htm}.

My students at Yonsei University were an important part of making the
first edition. One of the reasons that I got interested in using many of
the tools covered in this book, like using **knitr\} in slideshows, was
to improve a course I taught there: Introduction to Social Science Data
Analysis. I tested many of the explanations and examples in this book on
my students. Their feedback has been very helpful for making the book
clearer and more useful. Their experience with using these tools on
Microsoft Windows computers was also important for improving the book's
Windows documentation. Similarly, my students at the Hertie School of
Governance inspired and tested key sections of the second edition.

The vibrant community at Stack Overflow \url{http://stackoverflow.com/}
and Stack Exchange \url{http://stackexchange.com/} are always very
helpful for finding answers to problems that plague any computational
researcher. Importantly, the sites make it easy for others to find the
answers to questions that have already been asked.

My wife, Kristina Gandrud, has been immensely supportive and patient
with me throughout the writing of this book (and pretty much my entire
academic career). Certainly this is not the proper forum for musing
about marital relations, but I'll do a musing anyways. Having a person
who supports your interests, even if they don't completely share them,
is immensely helpful for a researcher. It keeps you going.

\hypertarget{additional-resources}{%
\chapter*{Additional Resources}\label{additional-resources}}


Additional resources that supplement the examples in this book can be
freely downloaded and experimented with. These resources include longer
examples discussed in individual chapters and a complete short
reproducible research project.

\hypertarget{chapter-examples}{%
\section*{Chapter Examples}\label{chapter-examples}}


Longer examples discussed in individual chapters, including files to
dynamically download data, code for creating figures, and markup files
for creating presentation documents, can be accessed at:
\textless{}\url{https://GitHub.com/christophergandrud/Rep-Res-Examples}\}.
Please see Chapter \ref{Storing} for more information on downloading
files from GitHub, where the examples are stored.\index{GitHub}

\hypertarget{short-example-project}{%
\section*{Short Example Project}\label{short-example-project}}


To download a full (though very short) example of a reproducible
research project created using the tools covered in this book go to:
\url{https://GitHub.com/christophergandrud/Rep-Res-ExampleProject1}.
Please follow the replication instructions in the main \emph{README.md}
file to fully replicate the project. It is probably a good idea to hold
off looking at this complete example in detail until after you have
become acquainted with the individual tools it uses. Become acquainted
with the tools by reading through this book and working with the
individual chapter examples.

The following two figures give you a sense of how the example's files
are organized. Figure \ref{ExampProjeFiles} shows how the files are
organized in the file system. Figure \ref{ExampProjDiagram} illustrates
how the main files are dynamically tied together. In the \emph{Data}
directory we have files to gather raw data from the \cite{WorldBank2013}
on fertilizer consumption and from \cite{Pemstein2010} on countries'
levels of democracy. They are tied to the data through the
\texttt{WDI()}\index{WDI()} and \texttt{download.file()}
functions.\index{R function!download.file()} A
\emph{Makefile}\index{Makefile} can run \emph{Gather1.R} and
\emph{Gather2.R} to gather and clean the data. It runs
\emph{MergeData.R} to merge the data into one data file called
\emph{MainData.csv}. It also automatically generates a variable
description file and a \emph{README.md}\index{README file} recording the
session info.\index{R!session info}

The \emph{Analysis} folder contains two files that create figures
presenting this data. They are tied to \emph{MainData.csv} with the
\texttt{read.csv()*\ function.\textbackslash{}index\{R\ function!read.csv\}\ These\ files\ are\ run\ by\ the\ presentation\ documents\ when\ they\ are\ knitted.\ The\ presentation\ documents\ tie\ to\ the\ analysis\ documents\ with\ *knitr*\ and\ the}source()``
function.\index{R function!source}

Though a simple example, hopefully these files will give you a complete
sense of how a reproducible research project can be organized. Please
feel free to experiment with different ways of organizing the files and
tying them together to make your research really reproducible.

\thispagestyle{plain}
    \begin{figure}[th!]
        \caption{Short Example Project File Tree}
        \label{ExampProjeFiles}
        \begin{center}
            %%%%%%%%%%%%%%
% Example research project file path
% Christopher Gandrud
% Updated 21 October 2014
%%%%%%%%%%%%%%

% Set node styles
\tikzstyle{DirBox} = [draw=black,
                      rectangle,
                      minimum width=5em,
                      very thick,
                      font=\small]

\tikzstyle{every node} = [draw=gray,
                          thin,
                          anchor=west,
                          font=\small]

% Begin tikz picture
\begin{tikzpicture}[%
  grow via three points={one child at (0.5,-0.7) and
  two children at (0.5,-0.7) and (0.5,-1.4)},
  edge from parent path={(\tikzparentnode.south) |- (\tikzchildnode.west)}]
  % Root Directory
  \node (root) at (5, 10) [DirBox]{Root};

  % Project Directory
  \node (project) at (4, 8.5) [DirBox]{Rep-Res-ExampleProject1}
        child {node {{\small{Paper.Rnw}}}}
        child {node {{\small{Slideshow.Rnw}}}}
        child {node {{\small{Website.Rnw}}}}
        child {node {{\small{Main.bib}}}}
            ;

  % Data Directory
  \node (data) at (0, 4.5) [DirBox]{Data}
      child {node {{\small{MainData.csv}}}}
      child {node {{\small{Makefile}}}}
      child {node {{\small{MergeData.R}}}}
      child {node {{\small{Gather1.R}}}}
      child {node {{\small{Gather2.R}}}}
      child {node {{\small{MainData\_VariableDescriptions.md}}}}
      child {node {{\small{README.Rmd}}}}
        ;

  % Analysis subdirectores/files
  \node (analysis) at (1.5, 7) [DirBox]{Analysis}
      child {node {{\small{GoogleVisMap.R}}}}
      child {node {{\small{ScatterUDSFert.R}}}}
        ;

  % README file
  \node (readme) at (9.5, 7) {README.md};

  % Connect boxes that are not explicit children
  \draw (root) -- (project);
  \draw (project) -| (analysis);
  \draw (analysis) -| (data);
  \draw (project) -| (readme);

\end{tikzpicture}

        \end{center}
    \end{figure}

\clearpage
\thispagestyle{plain}
\begin{landscape}
    \begin{figure}[th!]
        \caption{Short Example Main File Ties}
        \label{ExampProjDiagram}
        \begin{center}
            \input{images/additional_resources_images/example_diagram.tex}
        \end{center}
    \end{figure}
\end{landscape}

\hypertarget{about-the-author}{%
\chapter*{About the Author}\label{about-the-author}}


Christopher Gandrud is Economics Lead at Zalando SE building and
evaluating large scale decision-making systems. He was previously a
research fellow at the Institute for Quantitative Social Science,
Harvard University developing statistical software and applications for
the social and physical sciences. He has also held posts at City,
University of London, the Hertie School of Governance, Yonsei
University, and the London School of Economics where in 2012 he
completed a PhD in quantitative political science.

\hypertarget{setup}{%
\chapter*{Setup}\label{setup}}


Some setup is required to reproduce this book. Here are key R packages
you should consider installing and specific instructions for Windows and
Linux users.

\hypertarget{r-packages}{%
\section*{R Packages}\label{r-packages}}


In this book I discuss how to use a number of user-written R packages
for reproducible research. Many of these packages are not included in
the default R installation. They need to be installed separately.

\index{R!packages|(}

\textbf{Note:} in general you should aim to minimize the number of
packages that your research depends on. Doing so will lessen the
possibility that your code will ``break'' when a package is updated.
This book depends on relatively many packages because of its special and
unusual purpose of illustrating a variety of tools that you can use for
reproducible research.

To install key user-written packages discussed in this book, copy the
following code and paste it into your R console:

\begin{Shaded}
\begin{Highlighting}[]
\CommentTok{# Packages to install}
\NormalTok{pkg_to_install <-}\StringTok{ }\KeywordTok{c}\NormalTok{(}\StringTok{"brew"}\NormalTok{, }\StringTok{"bookdown"}\NormalTok{, }\StringTok{"xfun"}\NormalTok{)}


\CommentTok{# Check if the packages are installed, if not install}
\KeywordTok{lapply}\NormalTok{(pkg_to_install,}
    \ControlFlowTok{function}\NormalTok{(pkg) \{}
  \ControlFlowTok{if}\NormalTok{ (}\KeywordTok{system.file}\NormalTok{(}\DataTypeTok{package =}\NormalTok{ pkg) }\OperatorTok{==}\StringTok{ ''}\NormalTok{) }\KeywordTok{install.packages}\NormalTok{(pkg)}
\NormalTok{\})}
\end{Highlighting}
\end{Shaded}

\index{R!packages|(}

Once you enter this code, you may be asked to select a CRAN
``mirror''\index{CRAN!mirror} to download the packages from.\footnote{CRAN
  stands for the Comprehensive R Archive Network.} Simply select the
mirror closest to you.

\hypertarget{special-issues-for-windows-and-linux-users}{%
\section*{Special issues for Windows and Linux
Users}\label{special-issues-for-windows-and-linux-users}}
\addcontentsline{toc}{section}{Special issues for Windows and Linux
Users}

\index{Windows|(}

If you are using Windows, you will also need to install \emph{Rtools}
\cite[]{Rtools}.\index{Rtools} You can install \emph{Rtools} from:
\url{http://cran.r-project.org/bin/windows/Rtools/}.\label{RtoolsDownload}
Please use the recommended installation to ensure that your system
PATH\index{PATH} is set up correctly. Otherwise your computer will not
know where the tools are. Alternatively, use the
\texttt{install.Rtools()} function from the \emph{installr}
\citep{galili2018}\index{installr)} package to install it.

\index{Windows|(}

\index{Linux|(}

On Linux you will need to install the \emph{RCurl}
\citep{R-RCurl}\index{RCurl} and \emph{XML}\index{XML} \citep{R-XML}
packages separately. Use your Terminal\index{Terminal} to install these
packages with the following code:

\begin{Shaded}
\begin{Highlighting}[]
\FunctionTok{sudo}\NormalTok{ apt-get update}

\FunctionTok{sudo}\NormalTok{ apt-get install libcurl4-gnutls-dev}
\FunctionTok{sudo}\NormalTok{ apt-get install libxml2-dev}
\FunctionTok{sudo}\NormalTok{ apt-get install r-cran-xml}
\FunctionTok{sudo}\NormalTok{ apt-get install r-cran-rjava}
\end{Highlighting}
\end{Shaded}

\index{Linux|(}

\begin{Shaded}
\begin{Highlighting}[]
\NormalTok{xfun}\OperatorTok{::}\KeywordTok{session_info}\NormalTok{()}
\end{Highlighting}
\end{Shaded}

\begin{verbatim}
## R version 3.5.1 (2018-07-02)
## Platform: x86_64-apple-darwin15.6.0 (64-bit)
## Running under: macOS High Sierra 10.13.6
## 
## Locale: en_GB.UTF-8 / en_GB.UTF-8 / en_GB.UTF-8 / C / en_GB.UTF-8 / en_GB.UTF-8
## 
## Package version:
##   backports_1.1.2 base64enc_0.1.3 bookdown_0.7   
##   compiler_3.5.1  digest_0.6.17   evaluate_0.11  
##   glue_1.3.0      graphics_3.5.1  grDevices_3.5.1
##   highr_0.7       htmltools_0.3.6 jsonlite_1.5   
##   knitr_1.20      magrittr_1.5    markdown_0.8   
##   methods_3.5.1   mime_0.5        Rcpp_0.12.18   
##   rmarkdown_1.10  rprojroot_1.3-2 rstudioapi_0.7 
##   stats_3.5.1     stringi_1.2.4   stringr_1.3.1  
##   tinytex_0.8     tools_3.5.1     utils_3.5.1    
##   xfun_0.3        yaml_2.2.0
\end{verbatim}

\hypertarget{stylistic-convensions}{%
\chapter*{Stylistic Convensions}\label{stylistic-convensions}}


I use the following conventions throughout this book:

\begin{itemize}
\item
  \textbf{Abstract variables}: Abstract variables, i.e.~variables that
  do not represent specific objects in an example, are in
  \texttt{ALL\ CAPS\ TYPEWRITER\ TEXT}.
\item
  \textbf{Clickable buttons}: Clickable Buttons are in
  \texttt{typewriter\ text}.
\item
  \textbf{Code}: All code is in \texttt{typewriter\ text}.
\item
  \textbf{Filenames and directories}: Filenames and directories more
  generally are printed in \emph{italics}. I use CamelBack for file and
  directory names.
\item
  \textbf{File extensions}: Like filenames, file extensions are
  \emph{italicized}.
\item
  \textbf{Individual variable values}: Individual variable values
  mentioned in the text are in \emph{italics}.
\item
  \textbf{Objects}: Objects are printed in \emph{italics}. I use
  CamelBack for object names.
\item
  \textbf{Object columns}: Data frame object columns are printed in
  \emph{italics}.
\item
  Function names are followed by parentheses (e.g.,
  \texttt{stats::lm()})
\item
  \textbf{Packages}: \textbf{R} packages are printed in \emph{italics}.
\item
  \textbf{Windows and RStudio panes}: Open windows and RStudio panes are
  written in \emph{italics}.
\item
  \textbf{Variable names}: Variable names are printed in \textbf{bold}.
  I use CamelBack for individual variable names.
\end{itemize}

\mainmatter

\hypertarget{introduction}{%
\chapter{Introduction}\label{introduction}}

Now unplug your Internet cable, and start doing some serious work.

We have a nice figure in Figure \ref{fig:hello}, and also a table in
Table \ref{tab:iris}.

\begin{Shaded}
\begin{Highlighting}[]
\KeywordTok{par}\NormalTok{(}\DataTypeTok{mar =} \KeywordTok{c}\NormalTok{(}\DecValTok{4}\NormalTok{, }\DecValTok{4}\NormalTok{, }\DecValTok{1}\NormalTok{, }\FloatTok{.1}\NormalTok{))}
\KeywordTok{plot}\NormalTok{(cars, }\DataTypeTok{pch =} \DecValTok{19}\NormalTok{)}
\end{Highlighting}
\end{Shaded}

\begin{figure}
\includegraphics[width=0.9\linewidth]{bookdown_files/figure-latex/hello-1} \caption{Hello World!}\label{fig:hello}
\end{figure}

\begin{Shaded}
\begin{Highlighting}[]
\NormalTok{knitr}\OperatorTok{::}\KeywordTok{kable}\NormalTok{(}
  \KeywordTok{head}\NormalTok{(iris), }\DataTypeTok{caption =} \StringTok{'The boring iris data.'}\NormalTok{,}
  \DataTypeTok{booktabs =} \OtherTok{TRUE}
\NormalTok{)}
\end{Highlighting}
\end{Shaded}

\begin{table}

\caption{\label{tab:iris}The boring iris data.}
\centering
\begin{tabular}[t]{rrrrl}
\toprule
Sepal.Length & Sepal.Width & Petal.Length & Petal.Width & Species\\
\midrule
5.1 & 3.5 & 1.4 & 0.2 & setosa\\
4.9 & 3.0 & 1.4 & 0.2 & setosa\\
4.7 & 3.2 & 1.3 & 0.2 & setosa\\
4.6 & 3.1 & 1.5 & 0.2 & setosa\\
5.0 & 3.6 & 1.4 & 0.2 & setosa\\
5.4 & 3.9 & 1.7 & 0.4 & setosa\\
\bottomrule
\end{tabular}
\end{table}

More chapters to come in \texttt{02-foo.Rmd}, \texttt{03-bar}.Rmd,
\ldots{}

\hypertarget{the-foo-method}{%
\chapter{The FOO Method}\label{the-foo-method}}

We talk about the \emph{FOO} method\index{FOO} in this chapter.

\cleardoublepage

\hypertarget{appendix-appendix}{%
\appendix \addcontentsline{toc}{chapter}{\appendixname}}


\hypertarget{more-to-say}{%
\chapter{More to Say}\label{more-to-say}}

Yeah! I have finished my book, but I have more to say about some topics.
Let me explain them in this appendix.

To know more about \textbf{bookdown}, see \url{https://bookdown.org}.

\bibliography{book.bib,packages.bib}

\backmatter
\printindex

\end{document}
